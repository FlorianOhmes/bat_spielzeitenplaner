
Diese Bachelorarbeit dokumentiert die testgetriebene Entwicklung eines webbasierten 
Spielzeitenplaners für den Jugendfußball. Das Projekt wurde mit Spring Boot entwickelt und 
kombiniert den Ansatz des Test-Driven Development (TDD) mit den Prinzipien der 
Onion-Architektur. Es soll als ein Echtwelt-Beispiel für die TDD-Methode dienen. \\ 
Der Entwicklungsprozess umfasst die Ausarbeitung der Web-, Service- sowie der 
Persistenzschicht. 
Mithilfe von \texttt{MockMvcTests} und der Java-Bibliothek \texttt{Jsoup} ist jede einzelne 
Seite strukturiert entwickelt worden, von den einzelnen Bereichen über grundlegende 
UI-Komponenten, wie Formulare, bis hin zu dynamischen Inhalten, die mithilfe von Thymeleaf 
gerendert werden. Außerdem sind die Hauptaufgaben der Web-Steuereinheiten (Controller) 
mithilfe geeigneter Test überprüft worden. In der Service-Schicht sind Tests für zentrale 
Funktionen für das Speichern von Benutzeranfragen, das Bereitstellen von Daten sowie das 
Berechnen der Scores und Spielzeiten geschrieben worden. In der Persistenzschicht ist 
schließlich sichergestellt worden, dass das Speichern, Laden und Filtern der entsprechenden 
Entitäten ordnungsgemäß funktioniert. \\ 
Der Spielzeitenplaner richtet sich an Fußballlehrende im Amateur- bzw. im Jugendbereich 
und Sinn und Zweck der Anwendung ist es, eine faire und nachvollziehbare Verteilung 
der Spielminuten basierend auf strukturierten, individuellen Trainingsbewertungen zu 
ermöglichen. Außerdem soll er das Trainerteam dabei unterstützen, begründet und datenbasiert 
Entscheidungen bezüglich der Spielzeiten der Spieler zu treffen und diese transparent zu 
kommunizieren, um potenzielle Konflikte zwischen Spielern, Eltern und Trainerteam zu 
vermeiden. 

