%%%%%%%%%%%%%%%%%%%%%%%%%%%%%%%%%%%%%%%%%%%%%%%%%%%%%%%%%%%%%%%%%%%%%%%%%%%%%%%%
% Diese Datei beinhaltet den eigentlichen Inhalt Ihrer Arbeit.
%
% Es bietet sich der Übersicht halber an, die einzelnen Abschnitte jeweils
% in eigene Dateien zu schreiben und mittels \input einzubinden.
% Eine mögliche Verzeichnisstruktur sähe entsprechend so aus:
%
%     thesis/
%     +- tex/
%     |  +- introduction.tex
%     |  +- motivation.tex
%     |  +- experiments.tex
%     |  |  ...
%     |  +- conclusion.tex
%     +- abstract.tex
%     +- contents.tex
%     +- thesis.tex
%%%%%%%%%%%%%%%%%%%%%%%%%%%%%%%%%%%%%%%%%%%%%%%%%%%%%%%%%%%%%%%%%%%%%%%%%%%%%%%%





%%%%%%%%%%%%%%%%%%%%%%%%%%%%%%%%%%%%%%%%%%%%%%%%%%%%%%%%%%%%%%%%%%%%%%%%%%%%%%%%
% Einleitung
%%%%%%%%%%%%%%%%%%%%%%%%%%%%%%%%%%%%%%%%%%%%%%%%%%%%%%%%%%%%%%%%%%%%%%%%%%%%%%%%
\section{Einleitung}

Hier kommt die Einleitung hin. 

\pagebreak


%%%%%%%%%%%%%%%%%%%%%%%%%%%%%%%%%%%%%%%%%%%%%%%%%%%%%%%%%%%%%%%%%%%%%%%%%%%%%%%%
% Theoretische Grundlagen
%%%%%%%%%%%%%%%%%%%%%%%%%%%%%%%%%%%%%%%%%%%%%%%%%%%%%%%%%%%%%%%%%%%%%%%%%%%%%%%%
\section{Theoretische Grundlagen}

Dieser Arbeit sowie der Entwicklung des Projektes des Spielzeitenplaners liegen 
einige theoretische Konzepte, Methoden und Modelle zugrunde: zum einen der Ansatz des 
Test-Driven Development aus der Softwareentwicklung, zum anderen einige fußballerische 
Konzepte und Richtlinien, die in den folgenen Kapiteln kurz erläutert werden sollen. 
Außerdem richtet sich die Struktur des Projektes an der von Jeffrey Palermo in 2008 
vorgeschlagenen \texttt{Onion-Architektur} \cite{palermo2008onion}, die an dieser 
Stelle jedoch nicht explizit erläutert werden soll. 


\subsection{Die testgetriebene Entwicklung (TDD)}

Hier kommt der Theorieteil zu Test-driven Development hin. 


\subsection{Fußballerische Grundlagen und Konzepte}


Um Sinn und Zweck sowie die Funktionsweise der in dieser Arbeit vorliegenden Anwendung 
vollumfänglich verstehen zu können, sind neben softwaretechnischen Grundlagen auch das 
Wissen über grundlegende Konzepte des Fußballs sowie Richtlinien und Bestimmungen des 
Jugendfußballs notwendig. \\ 
Der Fußballverband Niederrhein -- kurz: FVN -- ist einer der 21 Verbände des Deutschen 
Fußballbundes (DFB). Er ist unter anderem für die Organisation eines geregelten 
Spielbetriebs im Amateurfußball am Niederrhein verantwortlich. Dies beinhaltet 
sämtliche Alters- aber auch Leistungsklassen im Senioren- und Juniorenbereich. Jedes 
Jahr werden auf der Webseite des FVN sogenannte Durchführungsbestimmungen 
veröffentlicht, die den Rahmen für die kommende Saison bilden \cite{fvn2024dufbest}. 
Dort ist beispielsweise die Dauer eines Fußballspiels für jede Altersklasse 
festgelegt. \\ 
Für die C-Jugend, die in der Saison 2024/2025 aus den Jahrgängen 2010 und 2011 
besteht, beträgt die Spielzeit insgesamt 70 -- eine Halbzeit also 35 -- Minuten. 
Gespielt wird mit elf Spielern pro Mannschaft, weitere fünf Spieler dürfen im Verlauf 
eines Spiels ein- und wieder ausgewechselt werden. Die elf Spieler einer Mannschaft, 
die zu Beginn des Spiels auf dem Platz stehen, bilden die sogenannte Startelf. Die 
restlichen Spieler werden auch als Reservespieler, Reserve oder einfach nur Bank -- in 
Anlehnung an die Sitzgelegenheit, auf der die Spieler Platz nehmen -- bezeichnet. \\ 
Startelf und Reservespieler bilden zusammen den Kader. Er setzt sich daher aus all 
denjenigen Spielern zusammen, die vom Trainerteam für ein Spiel nominiert werden, und 
kann von Spiel zu Spiel variieren, je nach Gesundheitszustand oder Trainingsstand der 
einzelnen Spieler oder aber aufgrund privater Termine der Akteure. Für weitere 
Bezeichnungen und allgemeine Fußball-Regeln sind die vom Deutschen Fußball-Bund 
veröffentlichten Fußball-Regeln zu studieren \cite{dfb2024regeln}. \\ 
Des Weiteren ist festzustellen, dass jede Mannschaft mit einer bestimmten Formation 
spielt. Die Formation spiegelt die räumliche Anordnung der Spieler auf dem Platz wider 
und hat zum Ziel, gewisse Symbiose-Effekte zwischen den einzelnen Spielern 
hervorzurufen sowie für eine ausgeglichene Aufteilung der Akteure auf dem Platz zu 
sorgen. Außerdem können auf Basis der gewählten Formation spezifische Taktiken gelehrt 
und angewendet werden, die für die hier vorliegende Arbeit jedoch nicht von Relevanz 
sind. \\ 
Beispiele für beliebte Formationen sind \texttt{4-2-3-1}, \texttt{4-3-3} oder 
\texttt{3-5-2}. Dabei werden im Namen die Anzahlen der Spieler nach Positionsgruppen 
sortiert und durch einen Bindestrich getrennt angegeben. \texttt{4-2-3-1} bedeutet 
also, dass die Abwehr aus vier Spielern -- der Viererkette -- besteht, der Sturm 
hingegen aus nur einem Spieler. Während sich die erste Zahl auf die Anzahl der 
Abwehrspieler bezieht und die letzte Zahl die Anzahl der Stürmer referenziert, bilden 
die restlichen Zahlen in der Mitte des Ausdrucks die Anzahl der Mittelfeldspieler, im 
Falle des \texttt{4-2-3-1} zwei defensive und drei offensive Mittelfeldspieler. Der 
Torwart bleibt bei der Bezeichnung einer Formation stets unerwähnt, da er immer 
vorhanden sein muss und immer nur aus einer Person besteht. \\ 
Innerhalb einer Formation nimmt jeder Spieler eine bestimmte Position ein. Eine 
Formation kann somit auch als eine Liste von elf Positionen interpretiert werden. 
Auch wenn im Kindesalter noch großer Wert auf eine ganzheitliche fußballerische 
Ausbildung gelegt wird, so ist es ab dem Jugendalter üblich, Spieler 
positionsspezifisch auszubilden. Jede Position bringt zum Teil sehr unterschiedliche 
Anforderungen mit sich, weshalb nicht jeder Spieler auf jeder Position spielen kann. 
Gängige -- oder grundlegende -- Positionen und ihre Bezeichnungen sind beispielsweise 
der Torwart (\texttt{TW}), der Innenverteidiger (\texttt{IV}), der linke/rechte 
Außenverteidiger (\texttt{LV/RV}), das zentrale defensive Mittelfeld (\texttt{ZDM}), 
das linke/rechte Mittelfeld (\texttt{LM/RM}), das zentrale offensive Mittelfeld 
(\texttt{ZOM}) und der Stürmer (\texttt{ST}). Positionsbezeichnungen werden 
üblicherweise in Großbuchstaben angegeben und versuchen, die Rolle der Position 
widerzuspiegeln. \\ 
Im Rahmen des Spielzeitenplaners haben Nutzende die Möglichkeit, eine eigene Formation 
und Positionen -- basierend auf den oben erläuterten Konventionen -- zu erstellen. 




%%%%%%%%%%%%%%%%%%%%%%%%%%%%%%%%%%%%%%%%%%%%%%%%%%%%%%%%%%%%%%%%%%%%%%%%%%%%%%%%
% Hauptteil
%%%%%%%%%%%%%%%%%%%%%%%%%%%%%%%%%%%%%%%%%%%%%%%%%%%%%%%%%%%%%%%%%%%%%%%%%%%%%%%%
\section{Die testgetriebene Entwicklung des Spielzeitenplaners}

Nachdem die theoretischen Grundlagen geklärt und erläutert worden sind, kann nun 
im folgenden Kapitel näher auf die testgetriebene Entwicklung des 
Spielzeitenplaners eingegangen werden. Dafür werden zunächst sein grundsätzlicher 
Aufbau und seine Funktionsweise skizziert, ehe im Anschluss das konkrete Testing 
in den unterschiedlichen Schichten der Architektur dokumentiert und verdeutlicht 
wird. 


\subsection{Aufbau und Funktionsweise des Spielzeitenplaners}


Der Spielzeitenplaner ist grundsätzlich in vier verschiedene Bereiche eingeteilt: 
Team, Recap, Spielzeiten planen und Einstellungen. Diese Bereiche und die damit 
verbundenen Grundfunktionen des Spielzeitenplaners sollen im Folgenden kurz 
beschrieben und erläutert werden. \\ 
Als Erstes ist der Team-Bereich zu nennen, der sich im Wesentlichen aus zwei Teilen 
zusammensetzt: der Team-Seite und der Spieler-Seite. Auf Ersterer lässt sich ein 
Teamname festlegen und speichern oder ändern. Außerdem wird eine Liste mit allen 
Spielern im Team und den dazugehörigen Daten (Name, Position, Trikotnummer, etc.) 
angezeigt. Für jeden Spieler gibt es die Möglichkeit, diesen entweder zu bearbeiten oder 
zu löschen. Mit einem Klick auf den Löschen-Button wird der entsprechende Spieler 
gelöscht, durch den Klick auf den Bearbeiten-Button hingegen gelangt man zur 
Spieler-Seite. \\ 
Diese enthält sowohl die individuellen Spieler-Daten des aktuell ausgewählten Spielers 
wie auch eine Anzeige der Spieler-Scores. Hier können Vor-, Nachname, Position und 
Trikotnummer des ausgewählten Spielers geändert werden. Wählt man auf der Team-Seite den 
Button zum Erstellen eines neuen Spielers, so wird die Spieler-Seite mit einem leeren 
Formular aufgerufen, sodass ein neuer Spieler erstellt und anschließend gespeichert 
werden kann. \\ 
Der Bereich Einstellungen enthält -- wie der Name bereits suggeriert -- einige 
grundsätzliche Einstellungen, die insbesondere für den Bereich des Planens der 
Spielzeiten von Bedeutung sind. Unter anderem besteht hier die Möglichkeit, eine 
eigene Formation zu erstellen. Dafür sind die Angabe eines Namens -- zum Beispiel 
\textit{4-2-3-1} oder \textit{4-3-3} -- sowie die Bezeichnungen der einzelnen 
Positionen notwendig. \\ 
Über dem Abschnitt zur Formation befindet sich der Kriterien-Abschnitt. Hier können 
Kriterien erstellt, bearbeitet und gelöscht werden. Diese sind von zentraler 
Bedeutung bei der Bewertung der Spieler im Recap-Bereich. Für die Erstellung eines 
Kriteriums wird ein Name bzw. eine Bezeichnung, eine Abkürzung (ein bis zwei 
Buchstaben) und eine Gewichtung benötigt. Die Summe aller Gewichte sollte stets 
eins ergeben, ein einzelnes Gewicht im Bereich zwischen null und eins liegen. 
Die wohl gebräuchlichsten Kriterien sind zum Beispiel die Trainingsbeteiligung und 
die Leistung. \\ 
Schließlich gibt es noch die Scores-Einstellungen, die ganz oben auf der Seite 
zu finden sind. In diesem Bereich lässt sich der Zeitraum festlegen, auf dessen 
Grundlage die Scores für die einzelnen Kriterien berechnet werden. Für das 
Kriterium der Trainingsbeteiligung gibt es zusätzlich noch besondere Einstellungen: Zum 
einen kann zwischen einer kurzfristigen und langfristigen Trainingsbeteiligung 
unterschieden, zum anderen können spezifische Gewichte für die Kurz- und Langfrist 
festgelegt werden. \\ 
Folgendes Beispiel soll zur Verdeutlichung des Sachverhaltes herangezogen werden: 
Ein Trainerteam entscheidet sich dazu, dass die Trainingsbeteiligung 50 Prozent 
des Gesamtscores ausmachen soll. Innerhalb der Trainingsbeteiligung wird dann 
nochmals festgelegt, dass die letzten drei Wochen für die Kurzfrist herangezogen 
werden sollen und die letzten acht Wochen für die Langfrist. Da das Trainerteam 
etwas mehr Wert auf eine langfristige Teilnahme am Training legt, werden die 
Gewichte auf $ 0.4 $ für die Kurzfrist und $ 0.6 $ für die Langfrist festgelegt. 
Somit berechnet sich der Score für dieses Kriterium zu 60 Prozent aus der 
langfristigen Trainingsbeteiligung und zu 40 Prozent aus der Kurzfrist. So kann 
ein Spieler, der grundsätzlich immer am Training teilnimmt, ein kurzfristiges 
Fehlen aufgrund von Krankheit oder schulischer Verpflichtungen durch einen hohen 
Score in der Langfrist korrigieren. \\ 
Der dritte große Bereich der Anwendung ist das Recap. Im Wesentlichen geschieht 
hier Folgendes: Nach jedem Training wird eine Bewertung eines jeden Spielers zu jedem 
Kriterium vorgenommen. Die Bewertungen werden gespeichert und zur Ermittlung der 
Scores für jedes Kriterium sowie des Gesamtscores herangezogen. Letzterer wiederum 
ist von wesentlicher Relevanz bei der Planung der Spielzeiten -- also der 
Spielminuten -- für das kommende Spiel. Um zur Recap-Seite zu gelangen, werden in 
einem ersten Schritt all diejenigen Spieler ausgewählt, die am Training 
teilgenommen haben. Diese Information wird dann auf der eigentlichen 
Bewertungsseite genutzt, um eine Vorsortierung der Spieler vorzunehmen. \\ 
Grundsätzlich ist die Recap-Seite nach den vorhandenen Kriterien 
gegliedert, das bedeutet, dass für jedes Kriterium eine Liste mit Spielern 
angezeigt wird, für die dann eine Bewertung abgegeben wird. Die Bewertung erfolgt 
auf einer Skala von eins bis fünf, wobei eine Drei den Durchschnitt bildet, eine 
Eins eine deutlich unterdurchschnittliche Bewertung darstellt und eine Fünf die 
bestmögliche Bewertung ist. Dementsprechend ist eine Vier als tendenziell 
überdurchschnittlich und eine Zwei als tendenziell unterdurchschnittlich zu 
betrachten. \\ 
Da der jeweilige Wert serverseitig als Double abgebildet wird, ist es den Nutzenden 
möglich, weitere Abstufungen vorzunehmen, beispielsweise eine $ 2.5 $ oder 
$ 4.5 $ zu vergeben. Für eine schnelle und effiziente Bewertung ist es jedoch 
ratsam, bei den Bewertungen eins, zwei, drei, vier und fünf zu bleiben. Für einen 
Spieler, der beim Training nicht anwesend ist, wird standardmäßig eine $ 0.0 $ 
vergeben. Die Null-Bewertungen werden dann serverseitig herausgefiltert und 
nicht gespeichert, da ein häufiges Fehlen sonst nicht nur den Score der 
Trainingsbeteiligung, sondern auch alle anderen Scores verringern würde, was einer 
gleich mehrfachen Abwertung gleichkäme. \\ 
Schließlich ist dann noch der Bereich der Spielzeitenplanung zu nennen. Hier können 
die Nutzenden vor einem Spiel planen, wie viele Minuten Spielzeit jeder einzelne 
Spieler basierend auf dem Gesamtscore erhalten soll. Die Spielzeitenplanung 
gestaltet sich als ein mehrstufiges Verfahren, durch das die Nutzenden vom 
Spielzeitenplaner geleitet werden. \\ 
In einem ersten Schritt werden zunächst alle Spieler ausgewählt, die für das 
kommende Spiel zur Verfügung stehen, die also nicht krank oder verletzt sind oder 
aufgrund von privaten Terminen und anderen Gründen abgesagt haben. Aus den 
verfügbaren Spielern wird dann in einem zweiten Schritt ermittelt, welche Spieler 
es in den Kader geschafft haben und welche nicht. Die vorgeschlagene Aufteilung 
kann dabei übernommen oder aber manuell durch die Nutzenden überarbeitet werden, 
sodass das Trainerteam die Kontrolle über die Spielzeitenplanung behält. \\ 
Nachdem der Kader feststeht und das entsprechende Formular durch die Benutzenden 
abgeschickt worden ist, wird in einem dritten Schritt aus dem Kader die Startelf 
bestimmt. Auch bei diesem Schritt können die Nutzenden Einfluss nehmen, indem 
Positionen getauscht oder Spieler von der Bank in die Startelf gesetzt werden. Ist 
die Startelf ermittelt, kommt es zum letzten Schritt der Spielzeitenplanung: 
dem Eintragen der Wechsel. Dieser letzte Planungsschritt ist essenziell für die 
Bestimmung der Spielzeiten, denn mit dem Feststehen der Wechsel bzw. der 
Wechsel-Zeitpunkte steht ebenfalls fest, welcher Spieler wie viele Minuten auf dem 
Platz steht. \\ 
Wie bei den vorherigen Planungsschritten besitzen die Nutzenden auch hier die volle 
Kontrolle: sowohl Einwechsel- wie Auswechselspieler aber auch die konkrete 
Spielminute kann bestimmt werden. Die voraussichtliche Anzahl der Spielminuten für 
jeden einzelnen Spieler wird auf Basis der gespeicherten Wechsel berechnet und auf 
der Seite angezeigt. Außerdem wird vom Spielzeitenplaner eine erwartete Spielzeit 
berechnet. Diese errechnet sich maßgeblich aufgrund des Gesamtscores eines Spielers 
und stellt diejenige Spielzeit dar, die basierend auf den Bewertungen des 
entsprechenden Spielers als fair erachtet wird. \\ 
Nun können Nutzende so lange wie nötig Anpassungen vornehmen -- das heißt neue 
Wechsel eintragen, Wechsel löschen oder die Wechselzeitpunkte anpassen -- bis 
die voraussichtliche Anzahl der Spielminuten eines jeden Spielers ungefähr mit der 
Anzahl der erwarteten Spielminuten übereinstimmt. Ein weiteres Mal ist es dem 
Trainerteam selbst überlassen zu entscheiden, ob erwartete und voraussichtliche 
Spielzeit beispielsweise bis auf fünf, zehn oder fünfzehn Minuten übereinstimmen 
müssen, dennoch sollten die beiden Kennzahlen so eng wie möglich beieinander 
liegen, da große Abweichungen die Sinnhaftigkeit die Spielzeitenplanung infrage 
stellen. \\ 
Ist schließlich eine faire Aufteilung der Spielzeiten unter allen beteiligten 
Akteuren gefunden, so kann die Spielzeitenplanung als erfolgreich abgeschlossen 
betrachtet werden und der Einsatz der Startelf sowie die Wechsel wie geplant 
durchgeführt werden. 




\subsection{Das systematische Testen der WebPages}


Nachdem Aufbau und Funktionsweise des Spielzeitenplaners beschrieben und erklärt 
worden sind, soll nun im Detail auf die testgetriebene Entwicklung der Anwendung 
eingegangen werden. Im Folgenden wird sich dabei zunächst auf das Testen des 
Web-Interfaces konzentriert. \\ 
Das Testing des Web-Interfaces ist grundsätzlich zweigeteilt: Zum einen werden die 
konkret ausgelieferten Webseiten mit ihren Inhalten und HTML-Elementen überprüft, 
zum anderen gibt es spezielle Testklassen für die Controller, mithilfe derer eine 
grundsätzliche Funktionsüberprüfung der Web-Steuereinheiten erfolgt. Die 
Testklassen für Ersteres sind im Verzeichnis \href{https://github.com/FlorianOhmes/bat_spielzeitenplaner/tree/main/spielzeitenplaner/src/test/java/de/bathesis/spielzeitenplaner/templates}{\texttt{de/bathesis/spielzeitenplaner/templates}} zu 
finden. Eine solche Testklasse ist grundsätzlich folgendermaßen aufgebaut: 

\begin{quote}
\begin{verbatim}
@WebMvcTest(Controller.class)
class PageTest {
    @Autowired
    MockMvc mvc;
    ...
    Document page;
    ...
    @BeforeEach
    void setUpPage() throws Exception {
        ...
        page = RequestHelper
                  .performGetAndParseWithJSoup(mvc, "/path");
    }
    ...
}
\end{verbatim}
\end{quote}

Durch die Verwendung der \texttt{WebMvcTest}-Annotation wird ein spezieller 
Testkontext initialisiert, der sich nur auf das Laden und Konfigurieren von 
Komponenten der Web-Schicht konzentriert \cite{vmware2024webmvctest}. 
Darüber hinaus wird in Klammern notiert, welche spezifische Controller-Klasse für 
den Test benötigt wird, sodass nur diese für den Test geladen und bereitgestellt 
wird. Überlicherweise enthält jede WebPage-Testklasse eine mit \texttt{BeforeEach} 
annotierte \texttt{setUp}-Methode, die vor jedem Test ausgeführt wird. Da das 
Vorgehen vor jedem Test gleich ist, bietet sich die Extraktion dieser Logik an, 
um die einzelnen Test übersichtlicher und wartbarer zu gestalten. \\ 
Das Ziel der \texttt{setUp}-Methode ist es, mithilfe des MockMvc eine Anfrage über 
einen bestimmten Pfad zu simulieren, das Ergebnis mit Jsoup zu parsen (siehe 
\href{https://github.com/FlorianOhmes/bat_spielzeitenplaner/blob/main/spielzeitenplaner/src/test/java/de/bathesis/spielzeitenplaner/utilities/RequestHelper.java}{\texttt{RequestHelper.java}}) und schließlich in einer 
Instanzvariable -- hier \texttt{page} genannt -- zu speichern. Auf diese Weise kann 
ein Abbild derjenigen Seite erzeugt werden, die im Browser zurückgegeben wird. 
Diese kann dann detaillierten Testungen unterzogen werden. \\ 
Ein wichtiger Bestandteil dabei bildet die Java-Bibliothek Jsoup. Sie wurde für das 
Arbeiten mit HTML entwickelt und ermöglicht daher sowohl das Parsen, wie auch 
Extrahieren, Manipulieren oder Korrigieren von HTML-Dokumenten 
\cite{hedley2024jsoup}. Für die dieser Arbeit zugrunde liegenden 
Testungen werden vor allem die erstgenannten Funktionen -- also das Parsen und 
Extrahieren von HTML bzw. HTML-Schnipseln -- benötigt. \\ 
Wie Jsoups \texttt{parse}-Funktion in diesem Projekt verwendet wurde, wurde bereits 
zuvor erklärt, das Extrahieren von HTML-Schnipseln lässt sich wie folgt 
realisieren: Mithilfe der \texttt{select}-Funktion wird aus einem HTML-Dokument 
oder einem HTML-Element ein HTML-Schnipsel herausgelöst, das den Anforderungen 
einer \texttt{CSS-Query} entspricht, die der Funktion als Parameter übergeben 
worden ist. \\ 
Über die Komplexität der \texttt{CSS-Query} kann gesteuert werden, wie 
detailliert die Struktur der HTML-Seite getestet werden soll. So kann mit 
\texttt{``h2''} zum Beispiel einfach eine Überschrift zweiter Ebene herausgefiltert 
werden, mit \texttt{``.card .card-body h2.card-title''} hingegen präziser nach 
einer \texttt{H2} gesucht werden, die die Klasse \texttt{card-title} besitzt und 
sich innerhalb des Bodies einer \texttt{Card} befindet. \\ 
Das Testing jeder einzelnen Webseite erfolgt im Wesentlichen nach ein und 
demselben Konzept, das im Folgenden geschildert werden soll. Zunächst werden die 
sogenannten Essentials -- oder auch Basics -- getestet, es wird also überprüft, ob 
die Seite die korrekte Überschrift besitzt sowie ob grundlegende Elemente -- wie 
die Navigationsleiste und der Footer -- angezeigt werden. Im Anschluss erfolgt eine 
detaillierte Überprüfung der einzelnen Bereiche der jeweiligen Seite und ihrer 
Elemente, also beispielsweise, ob der Bereich Teamname auf der Teamseite korrekt 
angezeigt wird oder ob das Formular zum Ändern des Teamnamens korrekt angezeigt 
wird (siehe \href{https://github.com/FlorianOhmes/bat_spielzeitenplaner/blob/main/spielzeitenplaner/src/test/java/de/bathesis/spielzeitenplaner/templates/team/TeamPageTest.java}{\texttt{TeamPageTest.java}}). \\ 
Abschließend wird getestet, ob entsprechende Daten, die durch den Controller bzw. 
das Model bereitgestellt werden, wie beabsichtigt mithilfe von Thymeleaf in die 
Seite gerendert werden. Durch diesen strukturierten Ansatz wird gewährleistet, dass 
alle wesentlichen Elemente und Funktionen der Webseite umfassend getestet 
werden. \\ 


Mit Beginn der Arbeiten an diesem Projekt wurden zunächst sämtliche HTML-Prototypen 
bzw. die ihnen zugrunde liegenden HTML-Templates testgetrieben entwickelt. Ein 
Beispiel dafür, auf das sich im Folgenden bezogen werden soll, ist die sogenannte 
\texttt{PlayerPage}, der das HTML-Template \href{https://github.com/FlorianOhmes/bat_spielzeitenplaner/blob/main/spielzeitenplaner/src/main/resources/templates/team/player.html}{\texttt{player.html}} als Basis dient, und die zugehörige 
Testklasse \href{https://github.com/FlorianOhmes/bat_spielzeitenplaner/blob/main/spielzeitenplaner/src/test/java/de/bathesis/spielzeitenplaner/templates/team/PlayerPageTest.java}{\texttt{PlayerPageTest.java}}. \\ 
Ein einfacher Test, durch den die Anwesenheit der korrekten H1-Überschrift 
erzwungen werden kann, ist \texttt{test\_01}: 

\begin{quote}
\begin{verbatim}
@Test
...
void test_01() {
    String expectedTitle = "Spieler bearbeiten/hinzufügen";
    String pageTitle = RequestHelper
              .extractTextFrom(playerPage, "h1");
    assertThat(pageTitle).isEqualTo(expectedTitle);
}
\end{verbatim}
\end{quote}

Hier wird mittels der Utilities-Klasse \href{https://github.com/FlorianOhmes/bat_spielzeitenplaner/blob/main/spielzeitenplaner/src/test/java/de/bathesis/spielzeitenplaner/utilities/RequestHelper.java}{\texttt{RequestHelper.java}} der 
Text des H1-Tags der \texttt{PlayerPage} extrahiert. Dann wird überprüft, ob dieser 
mit dem erwarteten Text übereinstimmt. Ist dies nicht der Fall, schlägt der Test 
fehl. Wenn auf der Seite überhaupt kein solches Element vorhanden ist, schlägt der 
Test ebenfalls fehl, denn dann ist das Ergebnis des Parsens durch Jsoup schlichtweg 
ein leerer String. Auf diese Weise kann also gewährleistet werden, dass auf einer 
Seite die korrekte Überschrift angezeigt wird. \\ 
Für die Navigationsleiste wurde auf der \texttt{PlayerPage} -- wie auch auf allen 
anderen Seiten -- getestet, ob diese vorhanden ist: 

\begin{quote}
\begin{verbatim}
@Test
... 
void test_02() {
    Elements navbar = RequestHelper.extractFrom(playerPage, "nav");
    assertThat(navbar).isNotEmpty();
}
\end{verbatim}
\end{quote}

Dafür wird das \texttt{nav}-Element der Seite extrahiert und durch eine geeignete 
Assertion sichergestellt, dass diese nicht leer ist. Analog zur Navigationsleiste 
kann ebenfalls mit dem Footer verfahren werden. Die eigentliche Überprüfung, ob 
die \texttt{Essentials} den Anforderungen entsprechen, ist ausgelagert und in der 
\texttt{FragmentsTest.java} unter 
\href{https://github.com/FlorianOhmes/bat_spielzeitenplaner/blob/main/spielzeitenplaner/src/test/java/de/bathesis/spielzeitenplaner/templates/fragments/FragmentsTest.java}{\texttt{src/test/java/de/ \linebreak bathesis/spielzeitenplaner/templates/fragments}} 
zu finden. \\ 
Dort wird im Detail getestet, ob die einzelnen Elemente korrekt strukturiert sind, 
sie den gewünschten Text enthalten und die erwarteten Funktionen unterstützen -- im 
Falle der Navigationsleiste beispielsweise, ob die Links zu den Unterseiten 
\texttt{Team}, \texttt{Recap}, \texttt{Spielzeiten planen} sowie 
\texttt{Einstellungen} funktionieren. Das entsprechende Fragment -- also der 
wiederkehrende HTML-Schnipsel einer Webseite, der ausgelagert worden ist -- ist in 
der \texttt{basics.html} unter \href{https://github.com/FlorianOhmes/bat_spielzeitenplaner/tree/main/spielzeitenplaner/src/main/resources/templates/fragments}{\texttt{src/main/resources/templates/fragments}} 
gespeichert. Dort wird es mittels des \texttt{th:fragment}-Tags als Solches 
gekennzeichnet und benannt. \\ 
Ist dies geschehen, so kann es im Folgenden dann wiederverwendet werden, indem es 
mithilfe der Nutzung des \texttt{th-replace}-Tags im HTML-Template und Thymeleaf in 
die entsprechende Seite hineingerendert wird. Ein konkretes Beispiel für die 
Verwendung eines \texttt{Thymeleaf-Fragments} in diesem Projekt ist der Footer: 

\begin{quote}
\begin{verbatim}
<div th:fragment="footer">
    <footer class="footer fixed-bottom">
        <p>
            &copy; 2024 SpielzeitenPlaner. Alle Rechte vorbehalten. 
        </p>
    </footer>
</div>
\end{verbatim}
\end{quote}

Er ist mit dem \texttt{th:fragment}-Tag versehen und als \texttt{``footer''}, 
benannt. Innerhalb der \href{https://github.com/FlorianOhmes/bat_spielzeitenplaner/blob/main/spielzeitenplaner/src/main/resources/templates/welcome.html}{\texttt{welcome.html}}
kann er dann einfach mithilfe des einzeiligen Codeschnipsels
\texttt{<div \linebreak th:replace=``\textasciitilde\{fragments/basics :: footer\}''></div>}
eingefügt werden. \\ 
Durch diese Herangehensweise müssen die Funktionalitäten der \texttt{Essentials} 
nicht auf jeder Seite explizit getestet werden -- bei zehn verschiedenen Seiten 
wären das immerhin 30 Tests, die aber stets nur das Gleiche testen würden -- 
sondern lediglich das Vorhandensein gewisser Elemente. Wenn sich nun eine 
Beschriftung ändert, die Struktur angepasst werden soll oder ein neuer Bereich -- 
zum Beispiel \texttt{Statistik} -- hinzukommt, müssen die entsprechenden Tests nur 
an einer Stelle angepasst werden, die Testklassen für die einzelnen Webseiten 
bleiben unberührt, da hier -- wie zuvor bereits beschrieben -- nur das 
Vorhandensein geprüft wird. \\ 
Navigationsleiste und Footer sind im Rahmen des Spielzeitenplaners als feste 
Bestandteile jeder Seite eingeplant, um Letzteren eine Rahmung zu geben und 
Nutzenden die Navigation durch die einzelnen Bereiche der Anwendung zu erleichtern. 
Für den Fall, dass Navigationsleiste oder Footer jedoch komplett entfernt werden 
sollen, kann über die Verwendung des \texttt{Thymeleaf Layout Dialect} nachgedacht 
werden. Dieser ermöglicht es, Layout-Templates zu erstellen und zu definieren, die 
dann wiederum von anderen Templates verwendet werden können 
\cite{borowiec2024layouts}. \\ 
So kann die grundsätzliche Struktur einer Seite von ihrem konkreten Inhalt getrennt 
werden, was die Modularisierbarkeit fördert und die Wiederverwendbarkeit erhöht. 
Für das konkrete Testing bedeutet dies, dass ein Layout-Template ein Mal gesondert 
getestet wird, das Testing der einzelnen Seiten sich vollkommen auf den Inhalt 
konzentrieren kann. \\ 
Doch wie bereits zuvor beschrieben ist nicht nur das Testen der \texttt{Essentials} 
ein wichtiger Bestandteil der \texttt{WebPages}-Tests, sondern auch die Überprüfung 
der einzelnen Bereiche einer Seite und ihrer Elemente. Den Kern der Spieler-Seite 
bilden die Bereiche \texttt{Spieler-Daten} und \texttt{Spieler-Scores}. Durch das 
Entwickeln dreier Tests, die stets einen anderen Aspekt überprüfen, kann ein 
solcher Bereich im Produktivcode erzwungen und Schritt für Schritt geformt werden, 
bis er schließlich seine gegenwärtige Form erreicht hat. \\ 
In einem ersten Schritt kann ein Test geschrieben werden, der die grundsätzliche 
Struktur des Bereichs festlegt: 

\begin{quote}
\begin{verbatim}
@Test
...
void test_04() {
    String expectedCardTitle = "Spieler-Daten";
    List<String> expectedAttributes = new ArrayList<>(List.of(
        "Vorname", "Nachname", "Trikotnummer", "Position"
    ));

    String cardTitle = RequestHelper.extractTextFrom(
        playerPage, ".card.player-data .card-body .card-title"
    );
    String playerInfo = RequestHelper.extractTextFrom(
        playerPage, ".card.player-data .card-body .player-info"
    );

    assertThat(cardTitle).isEqualTo(expectedCardTitle);
    assertThat(playerInfo).contains(expectedAttributes);
}
\end{verbatim}
\end{quote}

Um diesen Test bestehen zu lassen, ist die Etablierung der Grundstruktur in der 
\texttt{player.html} erforderlich, wie dem Commit 
\href{https://github.com/FlorianOhmes/bat_spielzeitenplaner/commit/5e52549a16792f66ea818041bc364e6b3e5ac219}{\texttt{5e52549}}
im Detail entnommen werden kann. \\ 
In einem zweiten Schritt kann dann die Anwesenheit des Spieler-Formulars erzwungen 
werden, das zum einen dazu dient, die Daten eines Spielers anzuzeigen, zum anderen 
aber auch das Ändern bereits gespeicherter Informationen unterstützt. Wie 
\texttt{test\_05} der 
\href{https://github.com/FlorianOhmes/bat_spielzeitenplaner/blob/main/spielzeitenplaner/src/test/java/de/bathesis/spielzeitenplaner/templates/team/PlayerPageTest.java}{\texttt{PlayerPageTest.java}}
zu entnehmen ist, wird dort geprüft, ob es ein Formular mit der ID 
\texttt{playerForm} gibt sowie ob dieses die für die Verarbeitung der Daten 
notwendigen Elemente -- wie Input-, Label-Felder und einen Button -- besitzt. 
Außerdem wird an dieser Stelle ebenfalls gefordert, dass das Formular die 
\texttt{Post}-Methode verwenden und die Anfrage über \texttt{/team/savePlayer}
verschickt werden soll sowie einen Button vom Typen \texttt{submit} enthalten muss, 
wie den folgenden Zeilen zu entnehmen ist: 

\begin{quote}
\begin{verbatim}
Elements playerForm = RequestHelper.extractFrom(playerPage, 
    "form#playerForm[method=\"post\"][action=\"/team/savePlayer\"]"
);
String buttonLabel = RequestHelper.extractTextFrom(playerForm, 
    "button[type=\"submit\"]"
);
\end{verbatim}
\end{quote}

Durch solche spezifischen \texttt{CSS-Queries} können präzise Anforderungen an das 
Formular gestellt werden und sichergestellt werden, dass die notwendigen 
Funktionalitäten korrekt implementiert werden. Im hier vorliegenden Fall der 
Änderung von Spielerdaten kann bzw. muss parallel im Controller-Testing eine 
entsprechende Route und Methodenunterstützung implementiert werden (siehe Kapitel 
3.2). \\ 
Nachdem nun die grundlegende Struktur des Spielerdaten-Bereichs etabliert und das 
Spieler-Formular vorhanden ist, kann abschließend in einem dritten Schritt die 
korrekte Anzeige der konkreten Spieler-Daten überprüft werden: 

\begin{quote}
\begin{verbatim}
@Test
...
void test_07() {
    List<String> expectedValues = new ArrayList<>(List.of(
        Integer.toString(player.getId()), 
        player.getFirstName(), player.getLastName(), 
        player.getPosition(), 
        Integer.toString(player.getJerseyNumber())
    ));

    List<String> values = RequestHelper.extractFrom(
        playerPage, "form#playerForm input"
    ).eachAttr("value");

    assertThat(values)
        .containsExactlyInAnyOrderElementsOf(expectedValues);
}
\end{verbatim}
\end{quote}

Die \texttt{expectedValues} entsprechen den Attributen des \texttt{player}, der als 
Instanzvariable in der Testklasse definiert ist. Mithilfe des 
\texttt{RequestHelpers} und Jsoups \texttt{eachAttr}-Methode können die tatsächlich 
angezeigten Werte der Input-Felder in einer Liste gespeichert und anschließend 
überprüft werden. Damit der Test jedoch ordnungsgemäß funktioniert, muss der mit 
\texttt{@MockBean} annotierte \texttt{playerService} noch konfiguriert werden. 
Mithilfe von 
\texttt{when(playerService.loadPlayer(player.getId())).thenReturn(player)} 
geschieht dies innerhalb der \texttt{setUp}-Methode entsprechend. \\ 
Um den Test nun bestehen zu lassen und die gewünschte Funktionalität zu 
implementieren, muss Folgendes geschehen: Die Spieler-Daten, die vom 
entsprechenden Service an den Controller weitergegeben und durch Letzteren im 
Model bereitgestellt werden, müssen mit dem jeweiligen Input-Feld verknüpft werden. 
Die Existenz jener Input-Felder wurde bereits im vorherigen Test gefordert, nicht 
aber ihr spezifischer Inhalt. Mithilfe von Thymeleaf lässt sich die entsprechenden 
Daten komfortabel in das jeweilige Template bzw. die jeweilige Seite hineinrendern: 

\begin{quote}
\begin{verbatim}
<form id="playerForm" ... th:object="${playerForm}">
    ...
    <input type="text" ... th:field="*{firstName}" ...>
    ...
    <input type="text" ... th:field="*{lastName}" ...>
    ...
</form>
\end{verbatim}
\end{quote}

Hier wurde \texttt{th:object} verwendet, um das Formular mit dem Formular-Objekt 
\texttt{playerForm} aus dem Model zu paaren. Thymeleafs \texttt{th:field} bindet 
außerdem jedes einzelne Input-Feld an das entsprechende Attribut, also 
\texttt{firstName}, \texttt{lastName}, \texttt{position} und \texttt{jerseyNumber}. 
Dies ist nicht nur für eine spätere Validierung und der damit verbundenen Ausgabe 
der Formular-Fehler von Vorteil und notwendig, sondern sorgt darüber hinaus dafür, 
dass die Daten des aktuellen Spielers auf der \texttt{PlayerPage} angezeigt 
werden. \\ 
Zusammengefasst lässt sich noch einmal sagen, dass sich durch gezieltes, 
umfangreiches Testing unterschiedlicher Aspekte die wesentlichen Funktionalitäten 
und Bausteine einer Webseite testgetrieben entwickeln lassen: Von der 
grundlegenden Strukur einer Seite bis hin zum Aufbau ihrer spezifischen Bereiche, 
die ihnen innewohnenden Elemente zur Datenerfassung, Interaktion und Kommunikation 
-- also zum Beispiel Formulare -- sowie die Anzeige konkreter und gespeicherter 
Daten und Inhalte. \\ 
Diese grundlegenden Testprinzipien lassen sich auf die Entwicklung jeder einzelnen 
Webseite übertragen und an ihre spezifischen Anforderungen anpassen, 
beispielsweise bei der Spielerbewertung auf der \texttt{RecapPage} (siehe 
\href{https://github.com/FlorianOhmes/bat_spielzeitenplaner/blob/main/spielzeitenplaner/src/test/java/de/bathesis/spielzeitenplaner/templates/recap/RecapPageTest.java}{{\texttt{RecapPageTest.java}}})
oder der Anzeige und Bestätigung des Kaders in der Spielzeitenplanung (siehe 
\href{https://github.com/FlorianOhmes/bat_spielzeitenplaner/blob/main/spielzeitenplaner/src/test/java/de/bathesis/spielzeitenplaner/templates/spielzeiten/KaderPageTest.java}{\texttt{KaderPageTest.java}}). 




\subsection{Controller-Tests: Das Überprüfen der Hauptaufgaben der Web-Steuereinheiten}


Eng verbunden mit mit dem Testing der \texttt{WebPages} ist auch die Überprüfung der 
Web-Steuereinheiten, also der entsprechenden Controller. Wie bereits im 
vorangegangenen Kapitel erwähnt, können konkrete Inhalte nur in die Seite eingefügt 
werden, wenn diese im Model vorhanden sind. Dies stellt unter anderem eine Aufgabe 
des Controllers dar. Doch neben der Datenaufbereitung und -bereitstellung ist er 
außerdem auch für die Verarbeitung von Benutzeranfragen, das Verwalten der 
Anwendungslogik, die Fehlerbehandlung und das grundlegende Routing zuständig. 
Alle soeben genannten Aufgaben sollen in diesem Kapitel unter dem Aspekt der 
testgetriebenen Entwicklung des Spielzeitenplaners eingehend beleuchtet werden. \\ 
Begonnen werden soll mit dem letzten Punkt -- dem grundlegenden Routing, das den 
Startpunkt sämtlicher Controller-Tests darstellt. Denn wie bereits in Kapitel 3.1 
festgehalten, sind sämtliche Testungen der Elemente und Strukturen einer 
ausgelieferten Webseite unter Gebrauch eines \texttt{MockMvc}-Objektes nur möglich, 
wenn zuvor ein entsprechendes Routing etabliert und ein geeignetes Request-Mapping 
stattgefunden hat. \\ 
Der wohl simpelste Test zur Überprüfung einer Route kann im Commit 
\href{https://github.com/FlorianOhmes/bat_spielzeitenplaner/commit/3aec5fe64e1b73cbaace8d56c9fb315b274ed0ad}{3aec5fe}
betrachtet werden. Alles, was zum Bestehen des Tests zur Erreichbarkeit der 
Team-Seite benötigt wird, ist eine mit \texttt{@Controller} annotierte Klasse und 
eine mit \texttt{@GetMapping(``/team'')} beschriftete Handler-Methode, die wiederum 
den Namen eines Templates zurückgibt -- in diesem Fall die \texttt{team.html}, die im 
Verzeichnis \texttt{src/main/ressources/templates} existieren muss. \\ 
Im weiteren Verlauf der Entwicklung des Projektes ist die 
\texttt{team()}-Handler-Methode dann in einen eigens für diesen Bereich angelegten 
\texttt{TeamController} ausgelagert worden. Des Weiteren ist mit der Einführung des 
\texttt{TeamService} das gewünschte Verhalten -- hier die Rückgabe des Team-
Objektes, das den Teamnamen enthält -- gemockt und eine Überprüfung des 
\texttt{view}-Namen ergänzt worden, sodass sich final der folgende Testablauf 
ergibt:

\begin{quote}
\begin{verbatim}
@Test
...
void test_01() throws Exception {
    when(teamService.load())
      .thenReturn(new Team(142, "Holstein Kiel"));

    RequestHelper.performGet(mvc, "/team")
                 .andExpect(status().isOk())
                 .andExpect(view().name("team/team"));
}
\end{verbatim}
\end{quote}

Sobald die entsprechende Seite erreichbar ist, kann mit der testgetriebenen 
Entwicklung ihrer Struktur -- wie in Kapitel 3.1 verdeutlicht -- begonnen werden. 
Neben dem grundlegenden Routing ist der Controller aber ebenfalls für die 
Aufbereitung und Bereitstellung der durch die Service-Schicht zur Verfügung 
gestellten Daten verantwortlich. Eine Überprüfung dieser Verantwortlichkeit lässt 
sich wie folgt realisieren: 

\begin{quote}
\begin{verbatim}
@Test
@DisplayName("Das Model für die Team-Seite ist korrekt befüllt.")
void test_02() throws Exception {
    // Erstellen eines Team-Objektes zu Testzwecken
    // Mocking des Team-Services
    
    // Erstellen einiger Test-Spieler
    // Mocking des Player-Services
    
    // Erstellen der Total-Scores & Mocking

    RequestHelper.performGet(mvc, "/team")
                 .andExpect(model().attribute("teamForm", teamForm))
                 .andExpect(model().attribute("players", players))
                 .andExpect(model().attribute(
                     "totalScores", totalScores
                 ));
}
\end{verbatim}
\end{quote}

Aus Gründen der Übersichtlichkeit ist hier auf eine vollständige Darstellung des 
Tests verzichtet worden, der genaue Wortlaut bzw. der genaue Code ist der 
\href{https://github.com/FlorianOhmes/bat_spielzeitenplaner/blob/main/spielzeitenplaner/src/test/java/de/bathesis/spielzeitenplaner/web/TeamControllerTest.java}{\texttt{TeamControllerTest.java}} 
zu entnehmen. Den Kern dieses Tests bilden der mithilfe des \texttt{MockMvcs} 
simulierte Get-Request und die spezifische Überprüfung der HTTP-Antwort durch 
\texttt{andExpect}. Durch das zuvor konfigurierte Mocking der Services 
erhalten Entwickelnde die Kontrolle über die zur Verfügung gestellten Daten. Unter 
Verwendung von \texttt{model().attribute(``attributeName'', ``expectedValue'')} 
kann dann gezielt gesteuert werden, welche Werte \texttt{teamForm}, 
\texttt{players} und \texttt{totalScores} annehmen sollen, denn nur wenn sie dem 
\texttt{expectedValue} entsprechen, wird der gegebene Test bestehen. Für 
\texttt{players} beispielsweise bedeutet das, dass das Attribut genau diejenige 
Liste von Spielern als Wert annehmen muss, die durch die 
\texttt{loadPlayers}-Methode des \texttt{PlayerService} zurückgegeben wird. \\ 
Neben dem Überprüfen der Model-Attribute hat \texttt{test\_02} aber auch eine 
sinnvolle Nebenwirkung: Durch die Art und Weise, wie er geschrieben ist, werden die 
Existenz eines \texttt{Team}-Objektes und einer \texttt{TeamForm} sowie ein 
\texttt{TeamMapper} gefordert, der für die Übersetzung zwischen Domänen- und 
Formular-Objekt zuständig ist. Diese Aufteilung fördert die Trennung der 
Verantwortlichkeiten -- Web-UI von Geschäftslogik -- und erhöht damit auch die 
Wartbarkeit des Codes, da zukünftige Änderungen am Team-Formular von der 
Geschäftslogik losgelöst durchgeführt werden können. \\ 
Wie bisher gezeigt fokussieren sich die beiden vorangegangenen Tests auf das 
Anfordern von Ressourcen auf dem Server mittels eines GET-Requests und die damit 
verbundene Datenaufbereitung und -bereitstellung. Im Gegensatz dazu steht der 
POST-Request, der für das Erstellen oder Verändern einer Ressource auf dem Server 
verantwortlich ist. \\ 
Konkret für die \texttt{TeamPage} bedeutet dies, dass nicht nur das Aufrufen der 
Team-Seite sowie die Anzeige des Teamnamens und der Spieler im Team eine wichtige 
Funktion darstellt, die der Spielzeitenplaner gewährleisten sollte, sondern auch 
die Möglichkeit, den Teamnamen zu bearbeiten und zu ändern, Spielerinformationen zu 
aktualisieren oder sich nicht mehr im Team befindliche Akteure zu löschen. Für die 
Unterstützung manipulierender Benutzeranfragen ist auch hier zunächst einmal die 
Etablierung einer grundlegenden Route vonnöten: 

\begin{quote}
\begin{verbatim}
@Test
@DisplayName("Es werden Post-Requests über /team/teamname akzeptiert.")
void test_05() throws Exception {
    mvc.perform(post("/team/teamname").param("name", "Spring Boot FC"))
       .andExpect(status().is3xxRedirection())
       .andExpect(view().name("redirect:/team"));
}
\end{verbatim}
\end{quote}

Das grundsätzliche Prinzip solcher Anfragen in diesem Projekt ist es, für jeden 
solcher Requests eine individuelle Route anzulegen, die dann von einer speziellen 
Handler-Methode eines zuständigen Controllers verarbeitet wird. Im Anschluss an 
eine erfolgreiche Anfrage wird dann auf eine Get-Route weitergeleitet. Im Falle der 
\texttt{TeamPage} bedeutet dies Folgendes: Analog zum \texttt{GetMapping} wird hier 
parallel zum Testcode eine mit \texttt{@PostMapping(``/teamname'')} annotierte 
Handler-Methode geschrieben, die \texttt{``redirect:/team''} retourniert. Letzteres 
stellt sicher, dass nach abgeschlossener Verarbeitung zur Team-Seite umgeleitet 
wird. \\ 
In einem zweiten Schritt muss dann sichergestellt werden, dass die zuständige 
Service-Methode korrekt aufgerufen wird: 

\begin{quote}
\begin{verbatim}
@Test
void test_06() throws Exception {
    Team team = new Team(null, "Spring Boot FC");
    mvc.perform(post("/team/teamname").param("name", team.name()));
    verify(teamService).save(team);
}
\end{verbatim}
\end{quote}

Eine wie zuvor beschriebene Überprüfung kann mithilfe von \texttt{Mockitos} 
\texttt{verify}-Methode realisiert werden. Durch die letzte Zeile des 
\texttt{test\_06} kann zum einen sichergestellt werden, dass die 
\texttt{save}-Methode des \texttt{TeamService} -- also die für das Speichern des 
Teamnamens zuständige Service-Methode -- durch den \texttt{TeamController} 
aufgerufen wird, zum anderen aber auch überprüft werden, ob diese mit dem 
richtigen Parameter aufgerufen wird. Letzteres ist besonders wichtig bei 
schichtbasierten Softwarearchitekturen -- wie der Onion-Architektur, um zu 
gewährleisten, dass Objekte zwischen den Schichten korrekt übergeben werden. \\ 
Für die \texttt{changeTeamName}-Methode des \texttt{TeamControllers} bedeutet dies 
konkret, dass gewährleistet wird, dass der Teamname korrekt in das 
\texttt{TeamForm}-Objekt integriert wird, dieses wiederum korrekt in ein 
\texttt{Team}-Objekt übersetzt wird und schließlich als Parameter an die 
\texttt{save}-Methode übergeben wird. \\ 
Doch neben Routing und Verwalten der Anwendungslogik ist ein Controller -- 
insbesondere bei POST-Requests -- auch noch für die Validierung der 
Benutzereingaben zuständig. Beim Teamnamen muss daher überprüft werden, dass das 
entsprechende Input-Feld nicht leer bzw. blank ist und eine Länge von 100 Zeichen 
nicht überschreitet, wie der 
\href{https://github.com/FlorianOhmes/bat_spielzeitenplaner/blob/main/spielzeitenplaner/src/main/java/de/bathesis/spielzeitenplaner/web/forms/TeamForm.java}{\texttt{TeamForm.java}}
zu entnehmen ist. Für ein weniger triviales und von daher interessanteres Beispiel 
kann abermals die \texttt{PlayerPage} herangezogen werden -- genauer gesagt das 
Formular zur Verwaltung der Spieler-Daten: Hier müssen Vorname, Nachname, Position 
und Trikotnummer geeignet validiert werden. Folgender Test der 
\href{https://github.com/FlorianOhmes/bat_spielzeitenplaner/blob/main/spielzeitenplaner/src/test/java/de/bathesis/spielzeitenplaner/web/TeamControllerTest.java}{\texttt{TeamControllerTest.java}}
soll veranschaulichen, wie eine solche Validierung erzwungen bzw. kontrolliert 
werden kann: 

\begin{quote}
\begin{verbatim}
@Test
...
void test_12() throws Exception {
    String html = mvc.perform(post("/team/savePlayer")
            // Parameter hinzufügen 
            ... 
        )
        .andExpect(model().attributeErrorCount("playerForm", 5))
        .andReturn().getResponse().getContentAsString();
    String html2 = mvc.perform(post("/team/savePlayer")
            // Parameter hinzufügen
            ... 
        )
        .andExpect(model().attributeErrorCount("playerForm", 1))
        .andReturn().getResponse().getContentAsString();

    Elements errors = Jsoup.parse(html).select(".error");
    Elements errors2 = Jsoup.parse(html2).select(".error");
    assertThat(errors).hasSize(4);
    assertThat(errors2).hasSize(1);
}
\end{verbatim}
\end{quote}

Der hier gezeigte Test ist nach folgendem Prinzip aufgebaut: Zunächst wird wieder 
ein entsprechender POST-Request simuliert, ehe der \texttt{AttributeErrorCount} des 
zu betrachtenden Objektes -- in diesem Fall die \texttt{PlayerForm} -- überprüft 
wird. Im Anschluss wird dann noch kontrolliert, ob potenzielle Fehlermeldungen auch 
tatsächlich auf der Seite angezeigt werden. Dazu wird die durch die Anfrage 
zurückgegebene Antwort mit \texttt{Jsoup} geparst sowie alle \texttt{div}-Container 
mit der Klasse \texttt{error} extrahiert und im Bezug auf ihre Größe inspiziert. 
Für das hier vorgestellte Beispiel werden zwei verschiedene Anfragen simuliert, 
die jeweils so konzipiert sind, dass sie möglichst viele Fehler abdecken sollen. \\ 
Der ersten Anfrage, die final im String mit dem Namen \texttt{html} gespeichert 
wird, werden somit jeweils ein leerer String als Parameter für den Vornamen, 
Nachnamen und die Position hinzugefügt. Damit der \texttt{AttributeErrorCount} 
für die \texttt{PlayerForm} nun die erwartete Größe besitzt, müssen gleich 
mehrere Anpassungen im Produktivcode vorgenommen werden: In der entsprechenden 
Handler-Methode des Controllers muss die \texttt{PlayerForm} mit \texttt{@Valid} 
annotiert werden, damit die Benutzereingabe auch wirklich validiert wird. Des 
Weiteren muss direkt nach der \texttt{PlayerForm} ein weiterer Parameter vom Typ 
\texttt{BindingResult} eingefügt werden, in dem das Ergebnis der auf die Eingabe 
angewendeten Validierung gespeichert und an das Formular-Objekt gebunden wird. \\ 
Diese Schritte alleine reichen jedoch nicht aus, um den Test bestehen zu lassen. 
Damit das \texttt{BindingResult} nicht einfach leer bleibt, muss die eigentliche 
Validierung noch konfiguriert werden. Dies geschieht innerhalb der Formular-Klasse 
\href{https://github.com/FlorianOhmes/bat_spielzeitenplaner/blob/main/spielzeitenplaner/src/main/java/de/bathesis/spielzeitenplaner/web/forms/PlayerForm.java}{\texttt{PlayerForm.java}} 
mithilfe geeigneter Validierungs-Annotationen. 
Dort wird festgelegt, dass Attribute wie der Vor- und Nachname oder die Position 
nicht blank sein dürfen sowie Letztere zwischen einem und fünf Zeichen lang sein 
und die Trikotnummer zwischen eins und 99 liegen muss. \\ 
Im Falle der ersten Anfrage -- die oben bereits beschrieben wurde -- werden also 
sämtliche \texttt{@NotBlank}-Annotation geprüft sowie die \texttt{@Size}-Annotation 
der Position und die \texttt{@NotNull}-Annotation der Trikotnummer. Schließlich 
wird dann noch in der Variable \texttt{error} gespeichert, ob für jedes Eingabefeld 
ein entsprechender Fehler-Container auf der Seite vorhanden ist und ein Fehler 
angezeigt wird -- die beiden Fehler bezüglich der Position werden dabei in ein und 
demselben Container angezeigt, da sie sich jeweils auf dasselbe Attribut 
beziehen. \\ 
Im Falle der zweiten Anfrage wird dann noch eine vollständige Benutzereingabe 
simuliert, jedoch wird hierbei eine Trikotnummer über 100 gewählt, um die 
\texttt{Max}-Validierung der \texttt{jerseyNumber} zu begutachten. Dementsprechend 
wird also ein \texttt{AttributeErrorCount} von eins erwartet sowie ein Container 
mit einer Fehlermeldung auf der \texttt{PlayerPage}. \\ 
Auf die hier gezeigte Weise kann also die Validierung von Benutzereingaben und die 
Fehlerausgabe testgetrieben entwickelt werden. Das grundlegende Prinzip lässt sich 
prinzipiell auf andere Formulare übertragen, sogar verschachtelte Formulare sind 
umsetzbar, wie in der 
\href{https://github.com/FlorianOhmes/bat_spielzeitenplaner/blob/main/spielzeitenplaner/src/main/java/de/bathesis/spielzeitenplaner/web/forms/RecapForm.java}{\texttt{RecapForm.java}} 
und der 
\href{https://github.com/FlorianOhmes/bat_spielzeitenplaner/blob/main/spielzeitenplaner/src/main/java/de/bathesis/spielzeitenplaner/web/forms/FormAssessment.java}{\texttt{FormAssessment.java}} 
gezeigt. Dort wird zum einen das Recap-Formular an sich validiert, das bedeutet 
das Datum des Recaps darf weder leer sein noch in der Zukunft liegen, aber auch 
die Liste der Bewertungen soll validiert werden. Für jedes \texttt{FormAssessment} 
ist daher zusätzlich noch festgelegt, dass der Wert nicht \texttt{null} sowie 
zwischen null und fünf liegen muss. 




\subsection{Das Testen der Service-Schicht}


Nachdem nun ausgiebig über das Testing des Web-Interfaces gesprochen wurde, soll 
sich das kommende Kapitel nun einer weiteren wichtigen Komponente, der Service-
Schicht, widmen. Diese kann unabhängig von den anderen Komponenten entwickelt werden. 
Im Falle des hier vorliegenden Projektes -- der Entwicklung des Spielzeitenplaners 
-- ist die Service-Schicht schrittweise aufgebaut worden. \\ 
Dabei wurde sich an verschiedenen \texttt{use cases} orientiert, die die 
funktionalen Anforderungen der Anwendung hervorheben und die wiederum eine 
Interaktion der Nutzenden mit der Web-Oberfläche als Startpunkt besitzen. Ein 
Beispiel für einen in diesem Projekt vorliegenden \texttt{use case} wäre 
beispielsweise das Löschen eines Spielers durch die Nutzenden. Durch einen Klick 
auf den \texttt{Löschen}-Button auf der \texttt{TeamPage} wird das entsprechende 
Formular an den Server geschickt. Dort wird es von einer entsprechend 
konfigurierten Handler-Methode eines Controllers in Empfang genommen und 
verarbeitet. Im Zuge dessen wird die entsprechende Service-Methode -- in diesem 
Fall die \texttt{deletePlayer}-Methode des \texttt{PlayerService} -- aufgerufen, 
die den Fall dann bearbeitet. Schließlich ruft diese dann eine entsprechende 
Repository-Methode auf, die wiederum für die Löschung der Spieler-Daten in der 
Datenbank zuständig ist. \\ 
Innerhalb dieser Aufruf-Kette fordert also gewissermaßen eine Komponente die 
Existenz einer anderen. Bezogen auf die Onion-Architektur bedeutet dies: die äußere 
Web-Schicht erwartet das Vorhandensein der inneren Schichten -- also der 
Service-Schicht und des Domain-Models -- und diese wiederum die Verfügbarkeit der 
(äußeren) Persistenz-Schicht. Auf diese Weise können sich Entwickelnde vom einen 
Ende der Software-Zwiebel hindurch zum anderen arbeiten. \\ 
Ein konkretes Beispiel aus der Entwicklung des Spielzeitenplaners ist dem Commit 
\href{https://github.com/FlorianOhmes/bat_spielzeitenplaner/commit/0a64ca3359402c06358b43cc41236ca24c2ec9cd#diff-c21f1f59588419db3d2efda09d4e20682f83653b4ef81ca9847d7458bc5b2f5f}{\texttt{0a64ca3}}
zu entnehmen. Durch die Implementierung der Funktionalitäten in der Web-Schicht -- 
siehe dazu die Änderungen an der \texttt{TeamControllerTest.java} -- wird unter 
anderem das Vorhandensein der \texttt{deletePlayer}-Methode und damit verbunden auch 
die Existenz des \texttt{PlayerService} an sich gefordert. Um den Test also bestehen 
zu lassen und schließlich committen zu können ist die Erstellung einer 
\texttt{PlayerService.java} und einer darin enthaltenen \texttt{deletePlayer}-Methode 
zwingend erforderlich. \\ 
Sobald die Service-Klasse vorhanden ist, kann sie auch testgetrieben entwickelt 
werden. Im Folgenden soll nun zunächst das Vorgehen bei den sogenannten 
Basisoperationen geschildert werden, ehe dann auf komplexere Testfälle eingegangen 
werden soll. Das zuvor genannte Beispiel für die \texttt{deletePlayer}-Methode lässt 
sich einfach und unkompliziert wie folgt testgetrieben entwickeln: 

\begin{quote}
\begin{verbatim}
@Test
...
void test_01() {
    Integer playerId = 17;
    playerService.deletePlayer(playerId);
    verify(playerRepository).deleteById(playerId);
}
\end{verbatim}
\end{quote}

Hier wird also überprüft, ob der \texttt{PlayerService} die Löschanfrage korrekt 
weiterverarbeitet. In diesem Fall bedeutet dies, dass diese an das zuständige 
Repository delegiert wird. Dabei ist besonders wichtig, dass das richtige Repository 
-- hier also das \texttt{PlayerRepository} -- verwendet wird und die entsprechende 
Methode mit dem aus der ursprünglichen Anfrage gesendeten Objekt aufgerufen wird. \\ 
Um diesen Test so schreiben zu können, muss das \texttt{PlayerRepository} zunächst 
mithilfe von \texttt{Mockito} gemockt werden, um die Kontrolle über sämtliche 
Aktionen des Repository zu erhalten, und der \texttt{PlayerService} mit dem soeben 
gemockten Repository ordnungsgemäß initialisiert werden, was wiederum einen 
geeigneten Konstruktor in der \texttt{PlayerService.java} voraussetzt. In diesem Zuge 
wird dann auch das Interface 
\href{https://github.com/FlorianOhmes/bat_spielzeitenplaner/blob/e4bc878ddf49753d522e2363f3b258093bab1d2f/spielzeitenplaner/src/main/java/de/bathesis/spielzeitenplaner/services/PlayerRepository.java}{\texttt{PlayerRepository.java}}
innerhalb der Service-Schicht etabliert und seine Implementierung, die mit 
\texttt{@Repository} annotierte 
\href{https://github.com/FlorianOhmes/bat_spielzeitenplaner/blob/e4bc878ddf49753d522e2363f3b258093bab1d2f/spielzeitenplaner/src/main/java/de/bathesis/spielzeitenplaner/database/PlayerRepositoryImpl.java}{\texttt{PlayerRepositoryImpl.java}}, 
in der Persistenzschicht. Ihre testgetriebene Entwicklung ist in Kapitel 3.5 
beschrieben. Sind die zuvor genannten Voraussetzungen implementiert, so kann der 
\texttt{deletePlayer}-Methode der Aufruf \texttt{playerRepository.deleteById(id)} 
hinzugefügt werden, wobei \texttt{id} als Argument entgegen genommen wird (siehe dazu 
\href{https://github.com/FlorianOhmes/bat_spielzeitenplaner/blob/e4bc878ddf49753d522e2363f3b258093bab1d2f/spielzeitenplaner/src/main/java/de/bathesis/spielzeitenplaner/services/PlayerService.java}{\texttt{PlayerService.java nach Commit e4bc878}}). \\ 
Die Methode zum Löschen eines Spielers ist ein Beispiel für eine Basisoperation, bei 
der eine Anfrage an die entsprechende Repository-Methode weitergeleitet wird. Hier 
steht also die Delegation der Aufgabe im Vordergrund, während das Repository sich 
um die Löschlogik kümmert. Für die Verifizierung der Übergabe des korrekten 
Parameters wurde \texttt{deleteById(playerId)} auf \texttt{verify(playerRepository)} 
aufgerufen. Dies ist in diesem Fall auch völlig ausreichend, sollten jedoch mehrere 
Methodenaufrufe mit unterschiedlichen Argumenten auftreten, komplexe Objekte -- wie 
beispielsweise ein Spieler mit seinen verschieden Attributen -- übergeben oder aber 
Eigenschaften innerhalb des Methodenaufrufs verändert werden, so ist der Gebrauch 
eines \texttt{ArgumentCaptors} durchaus sinnvoll, wie \texttt{test\_01} der 
\texttt{RecapServiceTest.java} veranschaulicht: 

\begin{quote}
\begin{verbatim}
@Test
@DisplayName("Die Bewertungen werden gespeichert.")
void test_01() {
    List<Assessment> assessments = new ArrayList<>(List.of(
            // Hier manuell erstellte Bewertungen hinzufügen 
    ));
    ArgumentCaptor<Assessment> assessmentCaptor = 
        ArgumentCaptor.forClass(Assessment.class);

    recapService.submitAssessments(assessments);

    verify(assessmentRepository, times(4))
        .save(assessmentCaptor.capture());

    List<Assessment> savedAssessments = 
        assessmentCaptor.getAllValues();

    assertThat(savedAssessments).isEqualTo(assessments);
}
\end{verbatim}
\end{quote}

In dem oben gezeigten Test wird der \texttt{ArgumentCaptor} dazu benutzt, die 
verschiedenen Bewertungen, mit der die \texttt{save}-Methode aufgerufen wird, 
einzufangen. Bevor dies mithilfe \texttt{assessmentCaptor.capture()} geschehen kann, 
muss die spezifische Klasse, für die der \texttt{Captor} programmiert ist, zunächst 
bei der Initialisierung des \texttt{assessmentCaptors} konfiguriert werden. Nach dem 
Erfassungsvorgang können alle gespeicherten \texttt{Assessments} dann abgerufen und 
in einer Liste gespeichert, um schließlich weiteren Überprüfungen mithilfe von 
\texttt{asserThat} unterzogen werden zu können. \\ 
Während bisher Gesagtes vor allen Dingen das Delegieren eines Aufrufs an die 
Persistenzschicht und die Überprüfung der Übergabe von Parametern in den Vordergrund 
stellt, soll nun im Folgenden auf die Überprüfung von Rückgabewerten eingegangen 
werden. Eine in diesem Projekt häufig getestete Methode ist die 
\texttt{load}-Methode, die Daten aus der Datenbank anfragt und diese dem Web-UI zur 
Verfügung stellt. Beispielsweise liefert die \texttt{loadCriteria}-Methode des 
\texttt{SettingsService} die aktuell gespeicherten Kriterien, \texttt{loadFormation} 
stellt die aktuell genutzte Formation bereit und \texttt{loadPlayers} gewährleistet 
die Versorgung der Webschicht mit den aktuellen Spielern im Team. Letztere wird wie folgt getestet: 

\begin{quote}
\begin{verbatim}
@Test
...
void test_02() {
    List<Player> players = TestObjectGenerator.generatePlayers();
    when(playerRepository.findAll()).thenReturn(players);

    List<Player> loadedPlayers = playerService.loadPlayers();

    verify(playerRepository).findAll();
    assertThat(loadedPlayers)
        .containsExactlyInAnyOrderElementsOf(players);
}
\end{verbatim}
\end{quote}

Nachdem der \texttt{TestObjectGenerator} eine Liste von (Test-)Spielern 
bereitgestellt hat, kann das \texttt{PlayerRepository} so konfiguriert werden, dass 
es diese Liste von Spielern zurückgibt. Sämtliche Repositories werden für die 
Service-Tests gemockt, um die Service-Schicht von der Datenbank losgelöst testen zu 
können. In dem folgenden \texttt{Act}-Schritt des Tests wird die 
\texttt{loadPlayers}-Methode dann ausgeführt und die Rückgabe in einer lokalen 
Variable gespeichert. So kann dann abschließend getestet werden, ob die 
\texttt{findAll}-Methode des \texttt{PlayerRepository} aufgerufen wurde -- die 
Überprüfung der Delegation eines Requests ist ja bereits aus den vorangegangenen 
Tests bekannt -- und ob die geladenen Spieler auch wirklich der gewünschten Liste 
entsprechen. Dabei wird \texttt{containsExactlyInAnyOrder} verwendet... 




\subsection{Realitätsnahe Datenbank-Tests mit Testcontainers}

Hier kommt das Testen der Datenbank-Schicht hin. 

\clearpage


%%%%%%%%%%%%%%%%%%%%%%%%%%%%%%%%%%%%%%%%%%%%%%%%%%%%%%%%%%%%%%%%%%%%%%%%%%%%%%%%
% Schlussteil
%%%%%%%%%%%%%%%%%%%%%%%%%%%%%%%%%%%%%%%%%%%%%%%%%%%%%%%%%%%%%%%%%%%%%%%%%%%%%%%%
\section{Fazit}

Hier kommt das Fazit hin. 

