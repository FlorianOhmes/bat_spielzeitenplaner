
Die Softwareentwicklung steht heutzutage vor vielfältigen Herausforderungen, da moderne 
Software eine Vielzahl an Anforderungen erfüllen muss. So sollte diese beispielsweise 
skalierbar, flexibel und sicher wie auch qualitativ hochwertig und wartbar sein. 
Darüber hinaus sollte sie auf die Wünsche der Kunden und Nutzenden zugeschnitten sein, 
denn schließlich sind diese die Hauptnutzenden und wollen durch sie einen Mehrwert 
erlangen. \\ 
Dabei hat sich Test-Driven Development (TDD) als eine methodische Herangehensweise an 
Softwareentwicklung herauskristallisiert. Sie bietet einen strukturierten und 
qualitätsorientierten Ansatz, der heutzutage immer häufiger praktiziert wird. 
Test-Driven Development kann zur Qualität und Robustheit moderner Software beitragen, 
indem Anforderungen präzise definiert und Fehler frühzeitig erkannt werden können. 
Außerdem erhöht es die Modularität und damit verbunden auch die Wartbarkeit von 
Software. \\ 
Die im Rahmen dieser Arbeit erschaffene Software -- der Spielzeitenplaner -- ist nach 
den Prinzipien des Test-Driven Development entwickelt worden und soll als 
Echtwelt-Beispiel für die Methode dienen. Diese Arbeit wiederum soll den Prozess der 
testgetriebenen Entwicklung des Spielzeitenplaners dokumentieren und TDD-Neulingen somit 
einige Anregungen bieten, wie moderne Software strukturiert entwickelt werden kann. 
Dabei werden an zahlreichen Stellen konkrete Tests gezeigt und deren Sinn und Zweck 
erläutert. \\ 
Doch was ist nun eigentlich der Spielzeitenplaner und welchen Nutzen soll er welcher 
Zielgruppe bringen? Der Spielzeitenplaner ist eine Softwarelösung für den Jugendfußball 
im Amateurbereich. Er soll Fußballlehrende dabei unterstützen, begründet Entscheidungen 
bezüglich der Spielzeiten der einzelnen Spieler zu treffen. \\ 
Nicht selten kommt es zwischen Trainerteam und Spielern und/oder Eltern zu Diskussionen 
über die Einsatzzeit eines Akteurs in einem Fußballspiel. Dabei gibt es eine Diskrepanz 
zwischen der Wahrnehmung des Spielers und seinen Eltern sowie der des Trainerteams. 
Erstere sind der Meinung, ihr Sohn hätte mehr Spielzeit verdient als er im Spiel 
tatsächlich bekommen hat und bringen demzufolge Gründe hervor, warum ihre Ansicht 
gerechtfertigt ist. Das Trainerteam hingegen vertritt unter Umständen eine andere 
Ansicht, da sie den Spieler über Wochen hinweg im Training gesehen und beurteilt 
haben und dementsprechend zu einem anderen Ergebnis kommen. \\ 
Mit Blick auf die zuvor beschriebene Diskussion über die Spielzeit eines Spielers 
stellen sich gleich mehrere Fragen: Wie viel Spielzeit hat jeder einzelne Spieler 
verdient? Wie viele Spielminuten können für einen bestimmten Spieler als \texttt{fair} 
erachtet werden? Und was bedeutet \texttt{fair} in diesem Zusammenhang überhaupt und 
wie lässt sich eine faire Aufteilung der Spielzeit ermitteln? \\ 
Diese Fragen können mit dem Spielzeitenplaner beantwortet werden. Mit ihm ist es 
möglich, die eigene Mannschaft im Bezug auf die Spielzeiten der einzelnen Spieler zu 
verwalten. Dazu können Fußballlehrende eigene Kriterien erstellen anhand derer 
Bewertungen durchgeführt werden sollen. Nach jedem Training wird jeder Spieler im 
Hinblick auf jedes einzelne Kriterium bewertet -- auch \texttt{Recap} genannt. 
Diese Bewertungen werden gespeichert und als Grundlage für die Berechnung eines 
Gesamt-Scores verwendet. An Spieltagen können Trainerteams dann die Spielzeiten 
planen, indem aufgrund der berechneten Scores Kader, Startelf und Wechsel bestimmt 
werden. \\ 
Das Erstellen von Kriterien und Bewerten der Spieler nach jedem Training fördert die 
Objektivität bei der Bestimmung der Spielzeiten sowie die Fähigkeit, begründet 
Entscheidungen zu treffen und diese zu vertreten. Darüber hinaus kann ein solch 
strukturierter Ansatz -- wenn Entscheidungen denn transparent kommuniziert und den 
Beteiligten erklärt werden -- zu mehr Verständnis auf Spieler- und Elternseite 
führen. Schließlich macht es die Situation für den Spieler greifbarer und er weiß, 
wie bzw. in welchen Bereichen er sich verbessern kann und muss, um auf mehr Spielzeit 
zu kommen. \\ 
In der hier vorliegenden Arbeit sollen nun zunächst einmal einige theoretische 
Grundlagen erläutert werden -- Test-Driven Development sowie fußballerische 
Grundlagen, ehe im Anschluss die testgetriebene Entwicklung des Spielzeitenplaners 
im Fokus steht. Dabei soll in jeder Architekturschicht -- Web, Service und Persistenz 
-- genau beschrieben und erläutert werden, wie dort testgetrieben entwickelt worden 
ist und konkrete Beispiele aus dem Projekt des Spielzeitenplaners gezeigt werden. 

