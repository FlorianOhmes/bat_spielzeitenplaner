
Eng verbunden mit mit dem Testing der \texttt{WebPages} ist auch die Überprüfung der 
Web-Steuereinheiten, also der entsprechenden Controller. Wie bereits im 
vorangegangenen Kapitel erwähnt, können konkrete Inhalte nur in die Seite eingefügt 
werden, wenn diese im Model vorhanden sind. Dies stellt unter anderem eine Aufgabe 
des Controllers dar. Doch neben der Datenaufbereitung und -bereitstellung ist er 
außerdem auch für die Verarbeitung von Benutzeranfragen, das Verwalten der 
Anwendungslogik, die Fehlerbehandlung und das grundlegende Routing zuständig. 
Alle soeben genannten Aufgaben sollen in diesem Kapitel unter dem Aspekt der 
testgetriebenen Entwicklung des Spielzeitenplaners eingehend beleuchtet werden. \\ 
Begonnen werden soll mit dem letzten Punkt -- dem grundlegenden Routing, das den 
Startpunkt sämtlicher Controller-Tests darstellt. Denn wie bereits in Kapitel 3.1 
festgehalten, sind sämtliche Testungen der Elemente und Strukturen einer 
ausgelieferten Webseite unter Gebrauch eines \texttt{MockMvc}-Objektes nur möglich, 
wenn zuvor ein entsprechendes Routing etabliert und ein geeignetes Request-Mapping 
stattgefunden hat. \\ 
Der wohl simpelste Test zur Überprüfung einer Route kann im Commit 
\href{https://github.com/FlorianOhmes/bat_spielzeitenplaner/commit/3aec5fe64e1b73cbaace8d56c9fb315b274ed0ad}{3aec5fe}
betrachtet werden. Alles, was zum Bestehen des Tests zur Erreichbarkeit der 
Team-Seite benötigt wird, ist eine mit \texttt{@Controller} annotierte Klasse und 
eine mit \texttt{@GetMapping(``/team'')} beschriftete Handler-Methode, die wiederum 
den Namen eines Templates zurückgibt -- in diesem Fall die \texttt{team.html}, die im 
Verzeichnis \texttt{src/main/ressources/templates} existieren muss. \\ 
Im weiteren Verlauf der Entwicklung des Projektes ist die 
\texttt{team()}-Handler-Methode dann in einen eigens für diesen Bereich angelegten 
\texttt{TeamController} ausgelagert worden. Des Weiteren ist mit der Einführung des 
\texttt{TeamService} das gewünschte Verhalten -- hier die Rückgabe des Team-
Objektes, das den Teamnamen enthält -- gemockt und eine Überprüfung des 
\texttt{view}-Namen ergänzt worden, sodass sich final der folgende Testablauf 
ergibt:

\begin{quote}
\begin{verbatim}
@Test
...
void test_01() throws Exception {
    when(teamService.load())
      .thenReturn(new Team(142, "Holstein Kiel"));

    RequestHelper.performGet(mvc, "/team")
                 .andExpect(status().isOk())
                 .andExpect(view().name("team/team"));
}
\end{verbatim}
\end{quote}

Sobald die entsprechende Seite erreichbar ist, kann mit der testgetriebenen 
Entwicklung ihrer Struktur -- wie in Kapitel 3.1 verdeutlicht -- begonnen werden. 
Neben dem grundlegenden Routing ist der Controller aber ebenfalls für die 
Aufbereitung und Bereitstellung der durch die Service-Schicht zur Verfügung 
gestellten Daten verantwortlich. Eine Überprüfung dieser Verantwortlichkeit lässt 
sich wie folgt realisieren: 

\begin{quote}
\begin{verbatim}
@Test
@DisplayName("Das Model für die Team-Seite ist korrekt befüllt.")
void test_02() throws Exception {
    // Erstellen eines Team-Objektes zu Testzwecken
    // Mocking des Team-Services
    
    // Erstellen einiger Test-Spieler
    // Mocking des Player-Services
    
    // Erstellen der Total-Scores & Mocking

    RequestHelper.performGet(mvc, "/team")
                 .andExpect(model().attribute("teamForm", teamForm))
                 .andExpect(model().attribute("players", players))
                 .andExpect(model().attribute(
                     "totalScores", totalScores
                 ));
}
\end{verbatim}
\end{quote}

Aus Gründen der Übersichtlichkeit ist hier auf eine vollständige Darstellung des 
Tests verzichtet worden, der genaue Wortlaut bzw. der genaue Code ist der 
\href{https://github.com/FlorianOhmes/bat_spielzeitenplaner/blob/main/spielzeitenplaner/src/test/java/de/bathesis/spielzeitenplaner/web/TeamControllerTest.java}{\texttt{TeamControllerTest.java}} 
zu entnehmen. Den Kern dieses Tests bilden der mithilfe des \texttt{MockMvcs} 
simulierte Get-Request und die spezifische Überprüfung der HTTP-Antwort durch 
\texttt{andExpect}. Durch das zuvor konfigurierte Mocking der Services 
erhalten Entwickelnde die Kontrolle über die zur Verfügung gestellten Daten. Unter 
Verwendung von \texttt{model().attribute(``attributeName'', ``expectedValue'')} 
kann dann gezielt gesteuert werden, welche Werte \texttt{teamForm}, 
\texttt{players} und \texttt{totalScores} annehmen sollen, denn nur wenn sie dem 
\texttt{expectedValue} entsprechen, wird der gegebene Test bestehen. Für 
\texttt{players} beispielsweise bedeutet das, dass das Attribut genau diejenige 
Liste von Spielern als Wert annehmen muss, die durch die 
\texttt{loadPlayers}-Methode des \texttt{PlayerService} zurückgegeben wird. \\ 
Des Weiteren wird im Rahmen der Entwicklung des Spielzeitenplaners zwischen Formular- 
und Domänen-Objekten unterschieden, eine Mapper-Klasse ist für die entsprechende 
Konvertierung von der einen in die andere Darstellung verantwortlich. Für den 
Teamnamen bedeutet dies, dass dieser aus dem Team-Objekt in die \texttt{TeamForm} 
überführt wird, sodass er schließlich im Model bereitgestellt werden kann. Diese 
Aufteilung fördert die Trennung der Verantwortlichkeiten -- Web-UI von Geschäftslogik -- 
und erhöht damit auch die Wartbarkeit des Codes, da zukünftige Änderungen am 
Team-Formular von der Geschäftslogik losgelöst durchgeführt werden können. \\ 
Wie bisher gezeigt fokussieren sich die beiden vorangegangenen Tests auf das 
Anfordern von Ressourcen auf dem Server mittels eines GET-Requests und die damit 
verbundene Datenaufbereitung und -bereitstellung. Im Gegensatz dazu steht der 
POST-Request, der für das Erstellen oder Verändern einer Ressource auf dem Server 
verantwortlich ist. \\ 
Konkret für die \texttt{TeamPage} bedeutet dies, dass nicht nur das Aufrufen der 
Team-Seite sowie die Anzeige des Teamnamens und der Spieler im Team eine wichtige 
Funktion darstellt, die der Spielzeitenplaner gewährleisten sollte, sondern auch 
die Möglichkeit, den Teamnamen zu bearbeiten und zu ändern, Spielerinformationen zu 
aktualisieren oder sich nicht mehr im Team befindliche Akteure zu löschen. Für die 
Unterstützung manipulierender Benutzeranfragen ist auch hier zunächst einmal die 
Etablierung einer grundlegenden Route vonnöten: 

\begin{quote}
\begin{verbatim}
@Test
@DisplayName("Es werden Post-Requests über /team/teamname akzeptiert.")
void test_05() throws Exception {
    mvc.perform(post("/team/teamname").param("name", "Spring Boot FC"))
       .andExpect(status().is3xxRedirection())
       .andExpect(view().name("redirect:/team"));
}
\end{verbatim}
\end{quote}

Das grundsätzliche Prinzip solcher Anfragen in diesem Projekt ist es, für jeden 
solcher Requests eine individuelle Route anzulegen, die dann von einer speziellen 
Handler-Methode eines zuständigen Controllers verarbeitet wird. Im Anschluss an 
eine erfolgreiche Anfrage wird dann auf eine Get-Route weitergeleitet. Im Falle der 
\texttt{TeamPage} bedeutet dies Folgendes: Analog zum \texttt{GetMapping} wird hier 
parallel zum Testcode eine mit \texttt{@PostMapping(``/teamname'')} annotierte 
Handler-Methode geschrieben, die \texttt{``redirect:/team''} retourniert. Letzteres 
stellt sicher, dass nach abgeschlossener Verarbeitung zur Team-Seite umgeleitet 
wird. \\ 
In einem zweiten Schritt muss dann sichergestellt werden, dass die zuständige 
Service-Methode korrekt aufgerufen wird: 

\begin{quote}
\begin{verbatim}
@Test
...
void test_06() throws Exception {
    String teamName = "Spring Boot FC";
    TeamForm teamForm = new TeamForm();
    teamForm.setName(teamName);

    mvc.perform(post("/team/teamname").param("name", teamName));

    verify(teamService).save(teamForm);
}
\end{verbatim}
\end{quote}

Eine wie zuvor beschriebene Überprüfung kann mithilfe von \texttt{Mockitos} 
\texttt{verify}-Methode realisiert werden. Durch die letzte Zeile des 
\texttt{test\_06} kann zum einen sichergestellt werden, dass die 
\texttt{save}-Methode des \texttt{TeamService} -- also die für das Speichern des 
Teamnamens zuständige Service-Methode -- durch den \texttt{TeamController} 
aufgerufen wird, zum anderen aber auch überprüft werden, ob diese mit dem 
richtigen Parameter aufgerufen wird. Letzteres ist besonders wichtig bei 
schichtbasierten Softwarearchitekturen -- wie der Onion-Architektur, um zu 
gewährleisten, dass Objekte zwischen den Schichten korrekt übergeben werden. \\ 
Für die \texttt{changeTeamName}-Methode des \texttt{TeamControllers} bedeutet dies 
konkret, dass gewährleistet wird, dass der Teamname korrekt in das 
\texttt{TeamForm}-Objekt integriert wird. Dies geschieht üblicherweise automatisch 
durch Spring, indem eine Instanz des \texttt{TeamForm}-Objekts als Parameter der Methode 
verwendet wird, sodass die Werte der Anfrage automatisch an das Formular-Objekt gebunden 
werden. Anschließend muss die \texttt{teamForm} als Parameter an die \texttt{save}-Methode des \texttt{TeamServices} übergeben werden, der sich dann wiederum um die 
Konvertierung zum Domänen-Objekt und das Speichern kümmern kann. \\ 
Doch neben Routing und Verwalten der Anwendungslogik ist ein Controller -- 
insbesondere bei POST-Requests -- auch noch für die Validierung der 
Benutzereingaben zuständig. Beim Teamnamen muss daher überprüft werden, dass das 
entsprechende Input-Feld nicht leer bzw. blank ist und eine Länge von 100 Zeichen 
nicht überschreitet, wie der 
\href{https://github.com/FlorianOhmes/bat_spielzeitenplaner/blob/main/spielzeitenplaner/src/main/java/de/bathesis/spielzeitenplaner/web/forms/TeamForm.java}{\texttt{TeamForm.java}}
zu entnehmen ist. \\ 
Für ein weiteres, ein wenig umfangreicheres Beispiel kann abermals die 
\texttt{PlayerPage} herangezogen werden -- genauer gesagt das Formular zur Verwaltung der 
Spieler-Daten: Hier müssen Vorname, Nachname, Position und Trikotnummer geeignet 
validiert werden. Folgender Test der 
\href{https://github.com/FlorianOhmes/bat_spielzeitenplaner/blob/main/spielzeitenplaner/src/test/java/de/bathesis/spielzeitenplaner/web/TeamControllerTest.java}{\texttt{TeamControllerTest.java}}
soll veranschaulichen, wie eine solche Validierung schrittweise erzwungen bzw. 
kontrolliert werden kann: 

\begin{quote}
\begin{verbatim}
@Test
...
void test_12() throws Exception {
    String response = mvc.perform(post("/team/savePlayer")
                                    .param("firstName", "")
                                    .param("lastName", "")
                                    .param("position", ""))
                         .andExpect(model()
                             .attributeErrorCount("playerForm", 5))
                         .andReturn()
                             .getResponse().getContentAsString();
    Elements errors = Jsoup.parse(response).select(".error");
    assertThat(errors).hasSize(4);
}
\end{verbatim}
\end{quote}

Der hier gezeigte Test ist nach folgendem Prinzip aufgebaut: Zunächst wird wieder 
ein POST-Request mit geeigneten Parametern simuliert, ehe der 
\texttt{AttributeErrorCount} des zu betrachtenden Objektes -- in diesem Fall die 
\texttt{PlayerForm} -- überprüft wird. Im Anschluss wird dann noch kontrolliert, ob 
potenzielle Fehlermeldungen auch tatsächlich auf der Seite angezeigt werden. Dazu wird 
die durch die Anfrage zurückgegebene Antwort mit \texttt{Jsoup} geparst sowie alle 
\texttt{div}-Container mit der Klasse \texttt{error} extrahiert und im Bezug auf ihre 
Größe inspiziert. \\ 
Den Basisfall stellt hier ein leeres Spieler-Formular dar. Der Anfrage werden somit 
jeweils ein leerer String als Parameter für den Vornamen, Nachnamen und die Position 
hinzugefügt. Damit der \texttt{AttributeErrorCount} für die \texttt{PlayerForm} nun die 
erwartete Größe besitzt, müssen die folgenden Anpassungen im Produktivcode vorgenommen 
werden: In der entsprechenden Handler-Methode des Controllers muss die 
\texttt{PlayerForm} mit \texttt{@Valid} annotiert werden, damit die Benutzereingabe auch 
wirklich validiert wird. Des Weiteren muss direkt nach der \texttt{PlayerForm} ein 
weiterer Parameter vom Typ \texttt{BindingResult} eingefügt werden, in dem das Ergebnis 
der auf die Eingabe angewendeten Validierung gespeichert und an das Formular-Objekt 
gebunden wird. \\ 
Diese Schritte alleine reichen jedoch nicht aus, um den Test bestehen zu lassen. 
Damit das \texttt{BindingResult} nicht einfach leer bleibt, muss die eigentliche 
Validierung noch konfiguriert werden. Dies geschieht innerhalb der Formular-Klasse 
\href{https://github.com/FlorianOhmes/bat_spielzeitenplaner/blob/main/spielzeitenplaner/src/main/java/de/bathesis/spielzeitenplaner/web/forms/PlayerForm.java}{\texttt{PlayerForm.java}} 
mithilfe geeigneter Validierungs-Annotationen. 
Dort wird festgelegt, dass Attribute wie der Vor- und Nachname oder die Position 
nicht blank sein dürfen sowie Letztere zwischen einem und fünf Zeichen lang sein 
und die Trikotnummer zwischen eins und 99 liegen muss. \\ 
Im oben gezeigten Basisfall mit einem leeren Spieler-Formular werden also bereits 
sämtliche \texttt{@NotBlank}-Annotation geprüft sowie die \texttt{@Size}-Annotation 
der Position und die \texttt{@NotNull}-Annotation der Trikotnummer. Im Anschluss wird 
dann noch in der Variable \texttt{error} gespeichert, ob für jedes Eingabefeld 
ein entsprechender Fehler-Container auf der Seite vorhanden ist und ein Fehler 
angezeigt wird -- die beiden Fehler bezüglich der Position werden dabei in demselben 
Container angezeigt, da sie sich jeweils auf dasselbe Attribut beziehen. \\ 
Im weiteren Verlauf der Entwicklung der Validierungsfunktionalitäten sind dann noch 
gezielt Anfragen simuliert worden, die die noch fehlenden Möglichkeiten abdecken: 
Eine Benutzeranfrage mit einer Trikotnummer kleiner eins sowie einer Nummer größer 99, 
wie \texttt{test\_13()} sowie \texttt{test\_14()} zu entnehmen ist. Diese gewährleisten, 
dass sowohl die \texttt{Max}- wie auch die \texttt{Min}-Validierung der 
\texttt{jerseyNumber} ordnungsgemäß implementiert sind. \\ 
Auf die hier gezeigte Weise kann also die Validierung von Benutzereingaben und die 
Fehlerausgabe testgetrieben entwickelt werden. Das grundlegende Prinzip lässt sich 
prinzipiell auf andere Formulare übertragen, sogar verschachtelte Formulare sind 
umsetzbar, wie in der 
\href{https://github.com/FlorianOhmes/bat_spielzeitenplaner/blob/main/spielzeitenplaner/src/main/java/de/bathesis/spielzeitenplaner/web/forms/RecapForm.java}{\texttt{RecapForm.java}} 
und der 
\href{https://github.com/FlorianOhmes/bat_spielzeitenplaner/blob/main/spielzeitenplaner/src/main/java/de/bathesis/spielzeitenplaner/web/forms/FormAssessment.java}{\texttt{FormAssessment.java}} 
gezeigt. Dort wird zum einen das Recap-Formular an sich validiert, das bedeutet 
das Datum des Recaps darf weder leer sein noch in der Zukunft liegen, aber auch 
die Liste der Bewertungen soll validiert werden. Für jedes \texttt{FormAssessment} 
ist daher zusätzlich noch festgelegt, dass der Wert nicht \texttt{null} sowie 
zwischen null und fünf liegen muss. 

