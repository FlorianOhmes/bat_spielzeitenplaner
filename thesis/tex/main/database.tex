
In den vorangegangenen Kapiteln wurde gezeigt, wie innerhalb der Web- und 
Service-Schicht testgetrieben entwickelt werden kann. Dabei wurde zunächst die 
Benutzeroberfläche entwickelt und damit verbunden auch die Web-Steuereinheiten -- 
die Controller. Diese rufen unter anderem die für die Benutzeranfragen zuständigen 
Service-Methoden auf. Deren testgetriebene Entwicklung erforderte wiederum die Existenz 
einiger Repositories und ihrer Methoden, die für das Laden der entsprechenden Daten aus 
der Datenbank verantwortlich sind. \\ 
Das folgende Kapitel soll sich demnach der testgetriebenen Entwicklung dieser 
Repositories -- und damit verbunden der Persistenzschicht im Allgemeinen -- widmen.
Im Allgemeinen ist zu sagen, dass für sämtliche Datenbank-Tests die 
\texttt{Testcontainers}-Bibliothek in Kombination mit \texttt{Docker} verwendet wird. 
Eine solche Konfiguration bietet gleich mehrere Vorteile: Erstens werden so 
realistische Tests mit einer echten Instanz der für das Projekt ausgewählten 
Datenbank ermöglicht. Die Test-Datenbank kann somit exakt so wie die 
Produktiv-Datenbank konfiguriert werden, wodurch möglichst realistische Bedingungen 
simuliert werden. \\ 
Zweitens sorgt die Verwendung von \texttt{Testcontainers} dafür, dass jeder Test in 
einer bereinigten Umgebung stattfindet, um zu verhindern, dass sie einander 
beeinflussen und das Ergebnis verfälschen. Drittens wird das Hoch- und Runterfahren 
sowie das Setup des \texttt{Docker-Containers} automatisiert und von 
\texttt{Testcontainers} übernommen. \\ 
Grundsätzlich sind alle Datenbank-Testklassen, die im Verzeichnis 
\href{https://github.com/FlorianOhmes/bat_spielzeitenplaner/tree/main/spielzeitenplaner/src/test/java/de/bathesis/spielzeitenplaner/database}{\texttt{src/test/java/de \linebreak /bathesis/spielzeitenplaner/database}}
zu finden sind, auf die gleiche Art und Weise aufgebaut. Zunächst einmal muss die 
jeweilige Testklasse mit drei verschiedenen Annotationen versehen werden: 
(1.) \texttt{@DataJdbcTest}, durch die Spring eine spezielle Testumgebung für die 
Persistenzschicht konfiguriert, der Web-Layer zum Beispiel wird komplett deaktiviert, 
um ressourcenschonender arbeiten zu können, (2.) 
\texttt{@AutoConfigureTestDatabase(replace = AutoConfigureTestDatabase.Replace.NONE)}, 
die dafür sorgt, dass Spring die Produktiv-Datenbank nicht automatisch durch eine 
\texttt{In-Memory}-Datenbank ersetzt, sondern die durch \texttt{Testcontainers} 
konfigurierte Test-Datenbank benutzt, und (3.) \texttt{@Testcontainers}, die für das 
Verwaltung der innerhalb der Klasse definierten Container zuständig ist 
(\texttt{Start}, \texttt{Stop} und \texttt{CleanUp}). 

