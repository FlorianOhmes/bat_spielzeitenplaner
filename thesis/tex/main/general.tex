
Der Spielzeiten-Planer ist grundsätzlich in vier verschiedene Bereiche eingeteilt: 
Team, Recap, Spielzeiten planen und Einstellungen. Diese Bereiche und die damit 
verbundenen Grundfunktionen des Spielzeiten-Planers sollen im Folgenden kurz 
beschrieben und erläutert werden. \\ 
Als Erstes ist der Team-Bereich zu nennen, der sich im Wesentlichen aus zwei Teilen 
zusammensetzt: der Team-Seite und der Spieler-Seite. Auf Ersterer lässt sich ein 
Teamname festlegen und speichern oder ändern. Außerdem wird eine Liste mit allen 
Spielern im Team und ihren Daten (Name, Position, Trikotnummer, etc.) angezeigt. 
Für jeden Spieler gibt es die Möglichkeit, diesen entweder zu löschen oder zu 
bearbeiten. Mit einem Klick auf den Löschen-Button wird der entsprechende Spieler 
gelöscht, durch den Klick auf den Bearbeiten-Button hingegen gelangt man zur 
Spieler-Seite. \\ 
Diese enthält sowohl die Spieler-Daten wie auch eine Anzeige der Spieler-Scores. 
Hier können Vor-, Nachname, Position und Trikotnummer des ausgewählten Spielers 
geändert werden. Wählt man auf der Team-Seite den Button zum Erstellen eines neuen 
Spielers, so wird die Spieler-Seite mit einem leeren Formular aufgerufen, sodass 
ein neuer Spieler erstellt und anschließend gespeichert werden kann. \\ 
Der Bereich Einstellungen enthält -- wie der Name bereits suggeriert -- einige 
grundsätzliche Einstellungen, die insbesondere für den Bereich zum Planen der 
Spielzeiten von Bedeutung sind. Unter anderem besteht hier die Möglichkeit, eine 
eigene Formation zu erstellen. Dafür sind die Angabe eines Namens -- zum Beispiel 
\textit{4-2-3-1} oder \textit{4-3-3} -- sowie die Bezeichnungen der einzelnen 
Positionen notwendig. \\ 
Über dem Abschnitt zur Formation befindet sich der Kriterien-Abschnitt. Hier können 
Kriterien erstellt, bearbeitet und gelöscht werden. Diese sind von zentraler 
Bedeutung bei der Bewertung der Spieler im Recap-Bereich. Für die Erstellung eines 
Kriteriums wird ein Name bzw. eine Bezeichnung, eine Abkürzung (ein bis zwei 
Buchstaben) und eine Gewichtung benötigt. Die Summe aller Gewichte sollte stets 
eins ergeben, ein einzelnes Gewicht im Bereich zwischen null und eins liegen. 
Die wohl gebräuchlichsten Kriterien sind zum Beispiel die Trainingsbeteiligung und 
die Leistung. \\ 
Schließlich gibt es noch die Scores-Einstellungen, die ganz oben auf der Seite 
zu finden sind. In diesem Bereich lässt sich der Zeitraum festlegen, auf dessen 
Grundlage die Scores für die einzelnen Kriterien berechnet werden. Für das 
Kriterium der Trainingsbeteiligung gibt es nochmal besondere Einstellungen: Zum 
einen kann zwischen einer kurzfristigen und langfristigen Trainingsbeteiligung 
unterschieden, zum anderen können spezifische Gewichte für die Kurz- und Langfrist 
festgelegt werden. \\ 
Folgendes Beispiel soll zur Verdeutlichung des Sachverhaltes herangezogen werden: 
Ein Trainerteam entscheidet sich dazu, dass die Trainingsbeteiligung 50 Prozent 
des Gesamtscores ausmachen soll. Innerhalb der Trainingsbeteiligung wird dann 
nochmals festgelegt, dass die letzten drei Wochen für die Kurzfrist herangezogen 
werden sollen und die letzten acht Wochen für die Langfrist. Da das Trainerteam 
etwas mehr Wert auf eine langfristige Teilnahme am Training legt, werden die 
Gewichte auf $ 0.4 $ für die Kurzfrist und $ 0.6 $ für die Langfrist festgelegt. 
Somit berechnet sich der Score für dieses Kriterium zu 60 Prozent aus der 
langfristigen Trainingsbeteiligung und zu 40 Prozent aus der Kurzfrist. So kann 
ein Spieler, der grundsätzlich immer am Training teilnimmt, ein kurzfristiges 
Fehlen aufgrund von Krankheit oder schulischer Verpflichtungen durch einen hohen 
Score in der Langfrist korrigieren. \\ 
Der dritte große Bereich der Anwendung ist das Recap. Im Wesentlichen geschieht 
hier Folgendes: Nach jedem Training wird eine Bewertung jedes Spielers zu jedem 
Kriterium vorgenommen. Die Bewertungen werden gespeichert und zur Ermittlung der 
Scores für jedes Kriterium sowie des Gesamtscores herangezogen. Letzterer wiederum 
ist von wesentlicher Relevanz bei der Planung der Spielzeiten -- also der 
Spielminuten -- für das kommende Spiel. Um zur Recap-Seite zu gelangen, werden in 
einem ersten Schritt all diejenigen Spieler ausgewählt, die am Training 
teilgenommen haben. Diese Information wird dann auf der eigentlichen 
Bewertungsseite genutzt, um eine Vorsortierung der Spieler vorzunehmen. \\ 
Grundsätzlich ist die Recap-Seite nach den vorhandenen Kriterien 
gegliedert, das bedeutet, dass für jedes Kriterium eine Liste mit Spielern 
angezeigt wird, für die dann eine Bewertung abgegeben wird. Die Bewertung erfolgt 
auf einer Skala von eins bis fünf, wobei eine Drei den Durchschnitt bildet, eine 
Eins eine deutlich unterdurchschnittliche Bewertung darstellt und eine Fünf die 
bestmögliche Bewertung ist. Dementsprechend ist eine Vier als tendenziell 
überdurchschnittlich und eine Zwei als tendenziell unterdurchschnittlich zu 
betrachten. \\ 
Da der jeweilige Wert serverseitig als Double abgebildet wird, ist es den Nutzenden 
möglich, weitere Abstufungen vorzunehmen, beispielsweise eine $ 2.5 $ oder 
$ 4.5 $ zu vergeben. Für eine schnelle und effiziente Bewertung ist es jedoch 
ratsam, bei den Bewertungen eins, zwei, drei, vier und fünf zu bleiben. Für einen 
Spieler, der beim Training nicht anwesend ist, wird standardmäßig eine $ 0.0 $ 
vergeben. Die Null-Bewertungen werden dann serverseitig herausgefiltert und 
nicht gespeichert, da ein häufiges Fehlen sonst nicht nur den Score der 
Trainingsbeteiligung, sondern auch alle anderen Scores verringern würde, was einer 
gleich mehrfachen Abwertung gleichkäme. \\ 
Schließlich ist dann noch der Bereich der Spielzeitenplanung zu nennen. Hier können 
die Nutzenden vor einem Spiel planen, wie viele Minuten Spielzeit jeder einzelne 
Spieler basierend auf dem Gesamtscore erhalten soll. Die Spielzeitenplanung 
gestaltet sich als ein mehrstufiges Verfahren, durch das die Nutzenden vom 
Spielzeitenplaner geleitet werden. \\ 
In einem ersten Schritt werden zunächst alle Spieler ausgewählt, die für das 
kommende Spiel zur Verfügung stehen, die also nicht krank oder verletzt sind oder 
aufgrund von privaten Terminen und anderen Gründen abgesagt haben. Aus den 
verfügbaren Spielern wird dann in einem zweiten Schritt ermittelt, welche Spieler 
es in den Kader geschafft haben und welche nicht. Die vorgeschlagene Aufteilung 
kann dabei übernommen oder aber manuell durch die Nutzenden überarbeitet werden, 
sodass das Trainerteam die Kontrolle über die Spielzeitenplanung behält. \\ 
Nachdem der Kader feststeht und das entsprechende Formular durch die Benutzenden 
abgeschickt worden ist, wird in einem dritten Schritt aus dem Kader die Startelf 
bestimmt. Auch bei diesem Schritt können die Nutzenden Einfluss nehmen, indem 
Positionen getauscht oder Spieler von der Bank in die Startelf gesetzt werden. Ist 
die Startelf ermittelt, kommt es zum letzten Schritt der Spielzeitenplanung: 
dem Eintragen der Wechsel. Dieser letzte Planungsschritt ist essenziell für die 
Bestimmung der Spielzeiten, denn mit dem Feststehen der Wechsel bzw. der 
Wechsel-Zeitpunkte steht ebenfalls fest, welcher Spieler wie viele Minuten auf dem 
Platz steht. \\ 
Wie bei den vorherigen Planungsschritten besitzen die Nutzenden auch hier die volle 
Kontrolle: sowohl Einwechsel- wie Auswechselspieler aber auch die konkrete 
Spielminute kann bestimmt werden. Die voraussichtliche Anzahl der Spielminuten für 
jeden einzelnen Spieler wird auf Basis der gespeicherten Wechsel berechnet und auf 
der Seite angezeigt. Außerdem wird vom Spielzeitenplaner eine erwartete Spielzeit 
berechnet. Diese errechnet sich maßgeblich aufgrund des Gesamtscores eines Spielers 
und stellt diejenige Spielzeit dar, die basierend auf den Bewertungen des 
entsprechenden Spielers als fair erachtet wird. \\ 
Nun können Nutzende so lange wie nötig Anpassungen vornehmen -- das heißt neue 
Wechsel eintragen, Wechsel löschen oder die Wechselzeitpunkte anpassen -- bis 
die voraussichtliche Anzahl der Spielminuten eines jeden Spielers ungefähr mit der 
Anzahl der erwarteten Spielminuten übereinstimmt. Ein weiteres Mal ist es dem 
Trainerteam selbst überlassen zu entscheiden, ob erwartete und voraussichtliche 
Spielzeit beispielsweise bis auf fünf, zehn oder fünfzehn Minuten übereinstimmen 
müssen, dennoch sollten die beiden Kennzahlen so eng wie möglich beieinander 
liegen, da große Abweichungen die Sinnhaftigkeit die Spielzeitenplanung infrage 
stellen. \\ 
Ist schließlich eine faire Aufteilung der Spielzeiten unter allen beteiligten 
Akteuren gefunden, so kann die Spielzeitenplanung als erfolgreich abgeschlossen 
betrachtet werden und der Einsatz der Startelf sowie die Wechsel wie geplant 
durchgeführt werden. 

