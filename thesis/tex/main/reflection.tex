
Im folgenden Kapitel soll sich kritisch mit dem Entwicklungsprozess des 
Spielzeitenplaners und der gewählten Methode des Test-Driven Developments 
auseinandergesetzt werden. \\ 
Zunächst ist festzuhalten, dass TDD ein mächtiges Werkzeug zur Entwicklung moderner 
Software ist: Mit ihr kann die Qualität des Codes verbessert und die Anfälligkeit für 
Fehler in der Software verringert werden. Die kleinen, testbaren Einheiten, die durch 
konsequentes Anwenden des TDD-Prinzips entstehen, sind sauber, modular, gut wartbar und 
dienen gleichzeitig als eine Art Dokumentation des Entwicklungsprozesses. \\ 
Doch Test-Driven Development bringt auch einige Einschränkungen und Stolpersteine mit 
sich, die auch im Rahmen der Entwicklung des Spielzeitenplaners deutlich geworden sind 
und im Folgenden thematisiert werden sollen. Als Erstes ist hier der Faktor Zeit zu 
nennen. Dieser ist nicht unerheblich und in der heutigen schnelllebigen Welt immer mehr 
von Relevanz. Test-driven Development benötigt schlichtweg mehr Zeit als die gewöhnliche Entwicklung -- insbesondere zu Beginn des Entwicklungsprozesses, da zunächst erst ein 
geeigneter Test formuliert werden muss, bevor dann die konkrete Implementierung 
realisiert werden kann. \\ 
Die Gegenpole TDD vs. gewöhnliche Entwicklung können relativ treffend mithilfe einer 
Analogie aus der Leichtathletik beschrieben werden: der 100-Meter-Lauf vs. Marathon. 
Während es beim 100-Meter-Lauf im Prinzip darum geht, so schnell wie möglich ans Ziel zu 
kommen, ist beim Marathon eine langfristige, beständige Strategie gefragt, um ans Ziel zu 
kommen und schließlich Erfolg zu haben. Anstatt den Wert auf Schnelligkeit zu legen, 
erfordert TDD eher ein moderates, beständiges Tempo: Ähnlich wie beim Marathon wird in 
kleinen, stetigen und zu bewältigen Schritten gearbeitet, bis man schließlich am Ziel 
angekommen ist. Außerdem ist es nicht das Ziel, schnell funktionierenden Code zu 
produzieren, sondern langfristig stabil und wartbare Software zu erschaffen. Darum kann 
der Fortschritt bei TDD manchmal langsamer erscheinen, wovon man sich als Entwickelnde 
nicht entmutigen lassen sollte. \\ 
Wenn der zuvor beschrieben Faktor Zeit jedoch nicht in ausreichendem Maße vorhanden ist, 
weil beispielsweise Deadlines unbedingt eingehalten werden müssen, oder der Fokus auf 
einer schnellen Fertigstellung ohne Beachtung der langfristigen Wartbarkeit liegt, sollte 
eher über eine andere Methode der Softwareentwicklung nachgedacht oder die Prioritäten 
überdacht werden. \\ 
Neben dem Faktor Zeit ist außerdem die Tatsache zu nennen, dass die testgetriebene 
Entwicklung einen \texttt{Mind Shift} der Entwickelnden erfordert, da der gewöhnliche 
Entwicklungsprozess auf den Kopf gestellt wird, indem die Tests zuerst geschrieben 
werden. Dieses Umdenken ist für TDD-Anfänger unter Umständen schwierig 
aufrechtzuerhalten, insbesondere wenn die Idee für die konkrete Implementierung einer 
Funktionalität bereits in den Köpfen der Entwickelnden vorhanden ist, während das 
Schreiben des Tests zugleich noch unklar ist. \\ 
Im Rahmen der Entwicklung des Spielzeitenplaners kam es insbesondere beim 
Entwickeln gewisser Funktionalitäten der Service-Schicht zu solchen Situationen, in denen 
sich das Schreiben von Tests schwierig gestaltete, während die Idee für den Produktivcode 
bereits vorhanden war. Hier bestand die Gefahr, dass das TDD-Prinzip eben doch wieder 
umgekehrt wird, indem der Produktivcode als Anhaltspunkt für das Schreiben eines Tests 
genommen wurde. \\ 
Weitere wichtige Aspekte, die eng mit den zuvor genannten Argumenten der Zeit und 
des Umdenkens verbunden sind, sind Geduld, Disziplin und Vertrauen. 
Test-driven Development benötigt Geduld, die bereits genannten kleinen Fortschritte 
beharrlich umzusetzen und nicht in alte Muster zurückzufallen. Des Weiteren ist für diese 
Methode ein hohes Maß an Disziplin notwendig, denn der \texttt{Red-Green-Refactor}-Zyklus 
muss stets eingehalten werden, es sind keine Abkürzungen erlaubt, auch wenn ein Test für 
eine bestimmte Funktionalität auf den ersten Blick trivial erscheint. \\ 
Darüber hinaus ist es in stressigen Situationen -- wie beispielsweise engen Deadlines 
oder unvorhergesehenen Verzögerungen im Entwicklungsablauf -- verlockend, den TDD-Prozess 
zu umgehen, auch hier ist Disziplin notwendig, um die Vorteile des Ansatzes auch unter 
Druck nutzen können. Auch im Rahmen des Entwicklungsprozesses des Spielzeitenplaners ist 
es zu Verzögerungen gekommen, weshalb insbesondere zum Ende der Entwicklung der 
Zeitdruck und damit verbunden auch der Druck nach einer schnellen Lösung zunahm. \\ 
Außerdem braucht die testgetriebene Entwicklung Vertrauen auf Seiten der Entwickelnden. 
Das Vertrauen, dass die Methode bei konsequenter Anwendung zu den gewünschten Ergebnissen 
führt, auch wenn diese zu einem gewissen Zeitpunkt in der Entwicklung noch nicht absehbar 
sind. Das Vertrauen und die Akzeptanz, dass Tests immer wieder und so lange fehlschlagen, 
bis alle zum Bestehen nötigen Details implementiert sind, da dies eben genau von TDD 
beabsichtigt ist. \\ 
Und schließlich ist das Vertrauen in TDD eben auch angebracht, wenn eine neue 
Abhängigkeit notwendig geworden ist und eingeführt wird. So mussten beim 
Spielzeitenplaner beispielsweise mit Einführung des \texttt{PlayerService} die 
entsprechenden Abhängigkeiten -- insbesondere in der Web-Schicht -- gemockt werden, um 
wieder zu einem konsistenten Gesamtzustand zu gelangen. Dabei fungierten die Tests als 
ein Warnsystem, das Alarm schlägt, weil bestehende Tests unter Umständen noch nicht auf 
die sich verändernde Architektur abgestimmt sind. Hier ist also Vertrauen in die Tests 
und damit verbunden in den TDD-Ansatz notwendig sowie die stetige Reflexion über die 
geschriebenen Tests und ihren Zweck. 

