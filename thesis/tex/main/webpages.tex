
Nachdem Aufbau und Funktionsweise des Spielzeitenplaners beschrieben und erklärt 
worden sind, soll nun im Detail auf die testgetriebene Entwicklung der Anwendung 
eingegangen werden. Im Folgenden wird sich dabei zunächst auf das Testen des 
Web-Interfaces konzentriert. \\ 
Das Testing des Web-Interfaces ist grundsätzlich zweigeteilt: Zum einen werden die 
konkret ausgelieferten Webseiten mit ihren Inhalten und HTML-Elementen überprüft, 
zum anderen gibt es spezielle Testklassen für die Controller, mithilfe derer eine 
grundsätzliche Funktionsüberprüfung der Web-Steuereinheiten erfolgt. Die 
Testklassen für Ersteres sind im Verzeichnis \href{https://github.com/FlorianOhmes/bat_spielzeitenplaner/tree/main/spielzeitenplaner/src/test/java/de/bathesis/spielzeitenplaner/templates}{\texttt{de/bathesis/spielzeitenplaner/templates}} zu 
finden. Eine solche Testklasse ist grundsätzlich folgendermaßen aufgebaut: 

\begin{quote}
\begin{verbatim}
@WebMvcTest(Controller.class)
class PageTest {
    @Autowired
    MockMvc mvc;
    ...
    Document page;
    ...
    @BeforeEach
    void setUpPage() throws Exception {
        ...
        page = RequestHelper
                  .performGetAndParseWithJSoup(mvc, "/path");
    }
    ...
}
\end{verbatim}
\end{quote}

Durch die Verwendung der \texttt{WebMvcTest}-Annotation wird ein spezieller 
Testkontext initialisiert, der sich nur auf das Laden und Konfigurieren von 
Komponenten der Web-Schicht konzentriert \cite{vmware2024webmvctest}. 
Darüber hinaus wird in Klammern notiert, welche spezifische Controller-Klasse für 
den Test benötigt wird, sodass nur diese für den Test geladen und bereitgestellt 
wird. Überlicherweise enthält jede WebPage-Testklasse eine mit \texttt{BeforeEach} 
annotierte \texttt{setUp}-Methode, die vor jedem Test ausgeführt wird. Da das 
Vorgehen vor jedem Test gleich ist, bietet sich die Extraktion dieser Logik an, 
um die einzelnen Test übersichtlicher und wartbarer zu gestalten. \\ 
Das Ziel der \texttt{setUp}-Methode ist es, mithilfe des MockMvc eine Anfrage über 
einen bestimmten Pfad zu simulieren, das Ergebnis mit Jsoup zu parsen (siehe 
\href{https://github.com/FlorianOhmes/bat_spielzeitenplaner/blob/main/spielzeitenplaner/src/test/java/de/bathesis/spielzeitenplaner/utilities/RequestHelper.java}{\texttt{RequestHelper.java}}) und schließlich in einer 
Instanzvariable -- hier \texttt{page} genannt -- zu speichern. Auf diese Weise kann 
ein Abbild derjenigen Seite erzeugt werden, die im Browser zurückgegeben wird. 
Diese kann dann detaillierten Testungen unterzogen werden. \\ 
Ein wichtiger Bestandteil dabei bildet die Java-Bibliothek Jsoup. Sie wurde für das 
Arbeiten mit HTML entwickelt und ermöglicht daher sowohl das Parsen, wie auch 
Extrahieren, Manipulieren oder Korrigieren von HTML-Dokumenten 
\cite{hedley2024jsoup}. Für die dieser Arbeit zugrunde liegenden 
Testungen werden vor allem die erstgenannten Funktionen -- also das Parsen und 
Extrahieren von HTML bzw. HTML-Schnipseln -- benötigt. \\ 
Wie Jsoups \texttt{parse}-Funktion in diesem Projekt verwendet wurde, wurde bereits 
zuvor erklärt, das Extrahieren von HTML-Schnipseln lässt sich wie folgt 
realisieren: Mithilfe der \texttt{select}-Funktion wird aus einem HTML-Dokument 
oder einem HTML-Element ein HTML-Schnipsel herausgelöst, das den Anforderungen 
einer \texttt{CSS-Query} entspricht, die der Funktion als Parameter übergeben 
worden ist. \\ 
Über die Komplexität der \texttt{CSS-Query} kann gesteuert werden, wie 
detailliert die Struktur der HTML-Seite getestet werden soll. So kann mit 
\texttt{``h2''} zum Beispiel einfach eine Überschrift zweiter Ebene herausgefiltert 
werden, mit \texttt{``.card .card-body h2.card-title''} hingegen präziser nach 
einer \texttt{H2} gesucht werden, die die Klasse \texttt{card-title} besitzt und 
sich innerhalb des Bodies einer \texttt{Card} befindet. \\ 
Das Testing jeder einzelnen Webseite erfolgt im Wesentlichen nach ein und 
demselben Konzept, das im Folgenden geschildert werden soll. Zunächst werden die 
sogenannten Essentials -- oder auch Basics -- getestet, es wird also überprüft, ob 
die Seite die korrekte Überschrift besitzt sowie ob grundlegende Elemente -- wie 
die Navigationsleiste und der Footer -- angezeigt werden. Im Anschluss erfolgt eine 
detaillierte Überprüfung der einzelnen Bereiche der jeweiligen Seite und ihrer 
Elemente, also beispielsweise, ob der Bereich Teamname auf der Teamseite korrekt 
angezeigt wird oder ob das Formular zum Ändern des Teamnamens korrekt angezeigt 
wird (siehe \href{https://github.com/FlorianOhmes/bat_spielzeitenplaner/blob/main/spielzeitenplaner/src/test/java/de/bathesis/spielzeitenplaner/templates/team/TeamPageTest.java}{\texttt{TeamPageTest.java}}). \\ 
Abschließend wird getestet, ob entsprechende Daten, die durch den Controller bzw. 
das Model bereitgestellt werden, wie beabsichtigt mithilfe von Thymeleaf in die 
Seite gerendert werden. Durch diesen strukturierten Ansatz wird gewährleistet, dass 
alle wesentlichen Elemente und Funktionen der Webseite umfassend getestet 
werden. \\ 


Mit Beginn der Arbeiten an diesem Projekt wurden zunächst sämtliche HTML-Prototypen 
bzw. die ihnen zugrunde liegenden HTML-Templates testgetrieben entwickelt. Ein 
Beispiel dafür, auf das sich im Folgenden bezogen werden soll, ist die sogenannte 
\texttt{PlayerPage}, der das HTML-Template \href{https://github.com/FlorianOhmes/bat_spielzeitenplaner/blob/main/spielzeitenplaner/src/main/resources/templates/team/player.html}{\texttt{player.html}} als Basis dient, und die zugehörige 
Testklasse \href{https://github.com/FlorianOhmes/bat_spielzeitenplaner/blob/main/spielzeitenplaner/src/test/java/de/bathesis/spielzeitenplaner/templates/team/PlayerPageTest.java}{\texttt{PlayerPageTest.java}}. \\ 
Ein einfacher Test, durch den die Anwesenheit der korrekten H1-Überschrift 
erzwungen werden kann, ist \texttt{test\_01}: 

\begin{quote}
\begin{verbatim}
@Test
...
void test_01() {
    String expectedTitle = "Spieler bearbeiten/hinzufügen";
    String pageTitle = RequestHelper
              .extractTextFrom(playerPage, "h1");
    assertThat(pageTitle).isEqualTo(expectedTitle);
}
\end{verbatim}
\end{quote}

Hier wird mittels der Utilities-Klasse \href{https://github.com/FlorianOhmes/bat_spielzeitenplaner/blob/main/spielzeitenplaner/src/test/java/de/bathesis/spielzeitenplaner/utilities/RequestHelper.java}{\texttt{RequestHelper.java}} der 
Text des H1-Tags der \texttt{PlayerPage} extrahiert. Dann wird überprüft, ob dieser 
mit dem erwarteten Text übereinstimmt. Ist dies nicht der Fall, schlägt der Test 
fehl. Wenn auf der Seite überhaupt kein solches Element vorhanden ist, schlägt der 
Test ebenfalls fehl, denn dann ist das Ergebnis des Parsens durch Jsoup schlichtweg 
ein leerer String. Auf diese Weise kann also gewährleistet werden, dass auf einer 
Seite die korrekte Überschrift angezeigt wird. \\ 
Für die Navigationsleiste wurde auf der \texttt{PlayerPage} -- wie auch auf allen 
anderen Seiten -- getestet, ob diese vorhanden ist: 

\begin{quote}
\begin{verbatim}
@Test
... 
void test_02() {
    Elements navbar = RequestHelper.extractFrom(playerPage, "nav");
    assertThat(navbar).isNotEmpty();
}
\end{verbatim}
\end{quote}

Dafür wird das \texttt{nav}-Element der Seite extrahiert und durch eine geeignete 
Assertion sichergestellt, dass diese nicht leer ist. Analog zur Navigationsleiste 
kann ebenfalls mit dem Footer verfahren werden. Die eigentliche Überprüfung, ob 
die \texttt{Essentials} den Anforderungen entsprechen, ist ausgelagert und in der 
\texttt{FragmentsTest.java} unter 
\href{https://github.com/FlorianOhmes/bat_spielzeitenplaner/blob/main/spielzeitenplaner/src/test/java/de/bathesis/spielzeitenplaner/templates/fragments/FragmentsTest.java}{\texttt{src/test/java/de/ \linebreak bathesis/spielzeitenplaner/templates/fragments}} 
zu finden. \\ 
Dort wird im Detail getestet, ob die einzelnen Elemente korrekt strukturiert sind, 
sie den gewünschten Text enthalten und die erwarteten Funktionen unterstützen -- im 
Falle der Navigationsleiste beispielsweise, ob die Links zu den Unterseiten 
\texttt{Team}, \texttt{Recap}, \texttt{Spielzeiten planen} sowie 
\texttt{Einstellungen} funktionieren. Das entsprechende Fragment -- also der 
wiederkehrende HTML-Schnipsel einer Webseite, der ausgelagert worden ist -- ist in 
der \texttt{basics.html} unter \href{https://github.com/FlorianOhmes/bat_spielzeitenplaner/tree/main/spielzeitenplaner/src/main/resources/templates/fragments}{\texttt{src/main/resources/templates/fragments}} 
gespeichert. Dort wird es mittels des \texttt{th:fragment}-Tags als Solches 
gekennzeichnet und benannt. \\ 
Ist dies geschehen, so kann es im Folgenden dann wiederverwendet werden, indem es 
mithilfe der Nutzung des \texttt{th-replace}-Tags im HTML-Template und Thymeleaf in 
die entsprechende Seite hineingerendert wird. Ein konkretes Beispiel für die 
Verwendung eines \texttt{Thymeleaf-Fragments} in diesem Projekt ist der Footer: 

\begin{quote}
\begin{verbatim}
<div th:fragment="footer">
    <footer class="footer fixed-bottom">
        <p>
            &copy; 2024 SpielzeitenPlaner. Alle Rechte vorbehalten. 
        </p>
    </footer>
</div>
\end{verbatim}
\end{quote}

Er ist mit dem \texttt{th:fragment}-Tag versehen und als \texttt{``footer''}, 
benannt. Innerhalb der \href{https://github.com/FlorianOhmes/bat_spielzeitenplaner/blob/main/spielzeitenplaner/src/main/resources/templates/welcome.html}{\texttt{welcome.html}}
kann er dann einfach mithilfe des einzeiligen Codeschnipsels
\texttt{<div \linebreak th:replace=``\textasciitilde\{fragments/basics :: footer\}''></div>}
eingefügt werden. \\ 
Durch diese Herangehensweise müssen die Funktionalitäten der \texttt{Essentials} 
nicht auf jeder Seite explizit getestet werden -- bei zehn verschiedenen Seiten 
wären das immerhin 30 Tests, die aber stets nur das Gleiche testen würden -- 
sondern lediglich das Vorhandensein gewisser Elemente. Wenn sich nun eine 
Beschriftung ändert, die Struktur angepasst werden soll oder ein neuer Bereich -- 
zum Beispiel \texttt{Statistik} -- hinzukommt, müssen die entsprechenden Tests nur 
an einer Stelle angepasst werden, die Testklassen für die einzelnen Webseiten 
bleiben unberührt, da hier -- wie zuvor bereits beschrieben -- nur das 
Vorhandensein geprüft wird. \\ 
Navigationsleiste und Footer sind im Rahmen des Spielzeitenplaners als feste 
Bestandteile jeder Seite eingeplant, um Letzteren eine Rahmung zu geben und 
Nutzenden die Navigation durch die einzelnen Bereiche der Anwendung zu erleichtern. 
Für den Fall, dass Navigationsleiste oder Footer jedoch komplett entfernt werden 
sollen, kann über die Verwendung des \texttt{Thymeleaf Layout Dialect} nachgedacht 
werden. Dieser ermöglicht es, Layout-Templates zu erstellen und zu definieren, die 
dann wiederum von anderen Templates verwendet werden können 
\cite{borowiec2024layouts}. \\ 
So kann die grundsätzliche Struktur einer Seite von ihrem konkreten Inhalt getrennt 
werden, was die Modularisierbarkeit fördert und die Wiederverwendbarkeit erhöht. 
Für das konkrete Testing bedeutet dies, dass ein Layout-Template ein Mal gesondert 
getestet wird, das Testing der einzelnen Seiten sich vollkommen auf den Inhalt 
konzentrieren kann. \\ 
Doch wie bereits zuvor beschrieben ist nicht nur das Testen der \texttt{Essentials} 
ein wichtiger Bestandteil der \texttt{WebPages}-Tests, sondern auch die Überprüfung 
der einzelnen Bereiche einer Seite und ihrer Elemente. Den Kern der Spieler-Seite 
bilden die Bereiche \texttt{Spieler-Daten} und \texttt{Spieler-Scores}. Durch das 
Entwickeln dreier Tests, die stets einen anderen Aspekt überprüfen, kann ein 
solcher Bereich im Produktivcode erzwungen und Schritt für Schritt geformt werden, 
bis er schließlich seine gegenwärtige Form erreicht hat. \\ 
In einem ersten Schritt kann ein Test geschrieben werden, der die grundsätzliche 
Struktur des Bereichs festlegt: 

\begin{quote}
\begin{verbatim}
@Test
...
void test_04() {
    String expectedCardTitle = "Spieler-Daten";
    List<String> expectedAttributes = new ArrayList<>(List.of(
        "Vorname", "Nachname", "Trikotnummer", "Position"
    ));

    String cardTitle = RequestHelper.extractTextFrom(
        playerPage, ".card.player-data .card-body .card-title"
    );
    String playerInfo = RequestHelper.extractTextFrom(
        playerPage, ".card.player-data .card-body .player-info"
    );

    assertThat(cardTitle).isEqualTo(expectedCardTitle);
    assertThat(playerInfo).contains(expectedAttributes);
}
\end{verbatim}
\end{quote}

Um diesen Test bestehen zu lassen, ist die Etablierung der Grundstruktur in der 
\texttt{player.html} erforderlich, wie dem Commit 
\href{https://github.com/FlorianOhmes/bat_spielzeitenplaner/commit/5e52549a16792f66ea818041bc364e6b3e5ac219}{\texttt{5e52549}}
im Detail entnommen werden kann. \\ 
In einem zweiten Schritt kann dann die Anwesenheit des Spieler-Formulars erzwungen 
werden, das zum einen dazu dient, die Daten eines Spielers anzuzeigen, zum anderen 
aber auch das Ändern bereits gespeicherter Informationen unterstützt. Wie 
\texttt{test\_05} der 
\href{https://github.com/FlorianOhmes/bat_spielzeitenplaner/blob/main/spielzeitenplaner/src/test/java/de/bathesis/spielzeitenplaner/templates/team/PlayerPageTest.java}{\texttt{PlayerPageTest.java}}
zu entnehmen ist, wird dort geprüft, ob es ein Formular mit der ID 
\texttt{playerForm} gibt sowie ob dieses die für die Verarbeitung der Daten 
notwendigen Elemente -- wie Input-, Label-Felder und einen Button -- besitzt. 
Außerdem wird an dieser Stelle ebenfalls gefordert, dass das Formular die 
\texttt{Post}-Methode verwenden und die Anfrage über \texttt{/team/savePlayer}
verschickt werden soll sowie einen Button vom Typen \texttt{submit} enthalten muss, 
wie den folgenden Zeilen zu entnehmen ist: 

\begin{quote}
\begin{verbatim}
Elements playerForm = RequestHelper.extractFrom(playerPage, 
    "form#playerForm[method=\"post\"][action=\"/team/savePlayer\"]"
);
String buttonLabel = RequestHelper.extractTextFrom(playerForm, 
    "button[type=\"submit\"]"
);
\end{verbatim}
\end{quote}

Durch solche spezifischen \texttt{CSS-Queries} können präzise Anforderungen an das 
Formular gestellt werden und sichergestellt werden, dass die notwendigen 
Funktionalitäten korrekt implementiert werden. Im hier vorliegenden Fall der 
Änderung von Spielerdaten kann bzw. muss parallel im Controller-Testing eine 
entsprechende Route und Methodenunterstützung implementiert werden (siehe Kapitel 
3.2). \\ 
Nachdem nun die grundlegende Struktur des Spielerdaten-Bereichs etabliert und das 
Spieler-Formular vorhanden ist, kann abschließend in einem dritten Schritt die 
korrekte Anzeige der konkreten Spieler-Daten überprüft werden: 

\begin{quote}
\begin{verbatim}
@Test
...
void test_07() {
    List<String> expectedValues = new ArrayList<>(List.of(
        Integer.toString(player.getId()), 
        player.getFirstName(), player.getLastName(), 
        player.getPosition(), 
        Integer.toString(player.getJerseyNumber())
    ));

    List<String> values = RequestHelper.extractFrom(
        playerPage, "form#playerForm input"
    ).eachAttr("value");

    assertThat(values)
        .containsExactlyInAnyOrderElementsOf(expectedValues);
}
\end{verbatim}
\end{quote}

Die \texttt{expectedValues} entsprechen den Attributen des \texttt{player}, der als 
Instanzvariable in der Testklasse definiert ist. Mithilfe des 
\texttt{RequestHelpers} und Jsoups \texttt{eachAttr}-Methode können die tatsächlich 
angezeigten Werte der Input-Felder in einer Liste gespeichert und anschließend 
überprüft werden. Damit der Test jedoch ordnungsgemäß funktioniert, muss der mit 
\texttt{@MockBean} annotierte \texttt{playerService} noch konfiguriert werden. 
Mithilfe von 
\texttt{when(playerService.loadPlayer(player.getId())).thenReturn(player)} 
geschieht dies innerhalb der \texttt{setUp}-Methode entsprechend. \\ 
Um den Test nun bestehen zu lassen und die gewünschte Funktionalität zu 
implementieren, muss Folgendes geschehen: Die Spieler-Daten, die vom 
entsprechenden Service an den Controller weitergegeben und durch Letzteren im 
Model bereitgestellt werden, müssen mit dem jeweiligen Input-Feld verknüpft werden. 
Die Existenz jener Input-Felder wurde bereits im vorherigen Test gefordert, nicht 
aber ihr spezifischer Inhalt. Mithilfe von Thymeleaf lässt sich die entsprechenden 
Daten komfortabel in das jeweilige Template bzw. die jeweilige Seite hineinrendern: 

\begin{quote}
\begin{verbatim}
<form id="playerForm" ... th:object="${playerForm}">
    ...
    <input type="text" ... th:field="*{firstName}" ...>
    ...
    <input type="text" ... th:field="*{lastName}" ...>
    ...
</form>
\end{verbatim}
\end{quote}

Hier wurde \texttt{th:object} verwendet, um das Formular mit dem Formular-Objekt 
\texttt{playerForm} aus dem Model zu paaren. Thymeleafs \texttt{th:field} bindet 
außerdem jedes einzelne Input-Feld an das entsprechende Attribut, also 
\texttt{firstName}, \texttt{lastName}, \texttt{position} und \texttt{jerseyNumber}. 
Dies ist nicht nur für eine spätere Validierung und der damit verbundenen Ausgabe 
der Formular-Fehler von Vorteil und notwendig, sondern sorgt darüber hinaus dafür, 
dass die Daten des aktuellen Spielers auf der \texttt{PlayerPage} angezeigt 
werden. \\ 
Zusammengefasst lässt sich noch einmal sagen, dass sich durch gezieltes, 
umfangreiches Testing unterschiedlicher Aspekte die wesentlichen Funktionalitäten 
und Bausteine einer Webseite testgetrieben entwickeln lassen: Von der 
grundlegenden Strukur einer Seite bis hin zum Aufbau ihrer spezifischen Bereiche, 
die ihnen innewohnenden Elemente zur Datenerfassung, Interaktion und Kommunikation 
-- also zum Beispiel Formulare -- sowie die Anzeige konkreter und gespeicherter 
Daten und Inhalte. \\ 
Diese grundlegenden Testprinzipien lassen sich auf die Entwicklung jeder einzelnen 
Webseite übertragen und an ihre spezifischen Anforderungen anpassen, 
beispielsweise bei der Spielerbewertung auf der \texttt{RecapPage} (siehe 
\href{https://github.com/FlorianOhmes/bat_spielzeitenplaner/blob/main/spielzeitenplaner/src/test/java/de/bathesis/spielzeitenplaner/templates/recap/RecapPageTest.java}{{\texttt{RecapPageTest.java}}})
oder der Anzeige und Bestätigung des Kaders in der Spielzeitenplanung (siehe 
\href{https://github.com/FlorianOhmes/bat_spielzeitenplaner/blob/main/spielzeitenplaner/src/test/java/de/bathesis/spielzeitenplaner/templates/spielzeiten/KaderPageTest.java}{\texttt{KaderPageTest.java}}). 

