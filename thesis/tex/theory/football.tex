
Um Sinn und Zweck sowie die Funktionsweise der in dieser Arbeit vorliegenden Anwendung 
vollumfänglich verstehen zu können, sind neben softwaretechnischen Grundlagen auch das 
Wissen über grundlegende Konzepte des Fußballs sowie Richtlinien und Bestimmungen des 
Jugendfußballs notwendig. \\ 
Der Fußballverband Niederrhein -- kurz: FVN -- ist einer der 21 Verbände des Deutschen 
Fußballbundes (DFB). Er ist unter anderem für die Organisation eines geregelten 
Spielbetriebs im Amateurfußball am Niederrhein verantwortlich. Dies beinhaltet 
sämtliche Alters- aber auch Leistungsklassen im Senioren- und Juniorenbereich. Jedes 
Jahr werden auf der Webseite des FVN sogenannte Durchführungsbestimmungen 
veröffentlicht, die den Rahmen für die kommende Saison bilden \cite{fvn2024dufbest}. 
Dort ist beispielsweise die Dauer eines Fußballspiels für jede Altersklasse 
festgelegt. \\ 
Für die C-Jugend, die in der Saison 2024/2025 aus den Jahrgängen 2010 und 2011 
besteht, beträgt die Spielzeit insgesamt 70 -- eine Halbzeit also 35 -- Minuten. 
Gespielt wird mit elf Spielern pro Mannschaft, weitere fünf Spieler dürfen im Verlauf 
eines Spiels ein- und wieder ausgewechselt werden. Die elf Spieler einer Mannschaft, 
die zu Beginn des Spiels auf dem Platz stehen, bilden die sogenannte Startelf. Die 
restlichen Spieler werden auch als Reservespieler, Reserve oder einfach nur Bank -- in 
Anlehnung an die Sitzgelegenheit, auf der die Spieler Platz nehmen -- bezeichnet. \\ 
Startelf und Reservespieler bilden zusammen den Kader. Er setzt sich daher aus all 
denjenigen Spielern zusammen, die vom Trainerteam für ein Spiel nominiert werden, und 
kann von Spiel zu Spiel variieren, je nach Gesundheitszustand oder Trainingsstand der 
einzelnen Spieler oder aber aufgrund privater Termine der Akteure. Für weitere 
Bezeichnungen und allgemeine Fußball-Regeln sind die vom Deutschen Fußball-Bund 
veröffentlichten Fußball-Regeln zu studieren \cite{dfb2024regeln}. \\ 
Des Weiteren ist festzustellen, dass jede Mannschaft mit einer bestimmten Formation 
spielt. Die Formation spiegelt die räumliche Anordnung der Spieler auf dem Platz wider 
und hat zum Ziel, gewisse Symbiose-Effekte zwischen den einzelnen Spielern 
hervorzurufen sowie für eine ausgeglichene Aufteilung der Akteure auf dem Platz zu 
sorgen. Außerdem können auf Basis der gewählten Formation spezifische Taktiken gelehrt 
und angewendet werden, die für die hier vorliegende Arbeit jedoch nicht von Relevanz 
sind. \\ 
Beispiele für beliebte Formationen sind \texttt{4-2-3-1}, \texttt{4-3-3} oder 
\texttt{3-5-2}. Dabei werden im Namen die Anzahlen der Spieler nach Positionsgruppen 
sortiert und durch einen Bindestrich getrennt angegeben. \texttt{4-2-3-1} bedeutet 
also, dass die Abwehr aus vier Spielern -- der Viererkette -- besteht, der Sturm 
hingegen aus nur einem Spieler. Während sich die erste Zahl auf die Anzahl der 
Abwehrspieler bezieht und die letzte Zahl die Anzahl der Stürmer referenziert, bilden 
die restlichen Zahlen in der Mitte des Ausdrucks die Anzahl der Mittelfeldspieler, im 
Falle des \texttt{4-2-3-1} zwei defensive und drei offensive Mittelfeldspieler. Der 
Torwart bleibt bei der Bezeichnung einer Formation stets unerwähnt, da er immer 
vorhanden sein muss und immer nur aus einer Person besteht. \\ 
Innerhalb einer Formation nimmt jeder Spieler eine bestimmte Position ein. Eine 
Formation kann somit auch als eine Liste von elf Positionen interpretiert werden. 
Auch wenn im Kindesalter noch großer Wert auf eine ganzheitliche fußballerische 
Ausbildung gelegt wird, so ist es ab dem Jugendalter üblich, Spieler 
positionsspezifisch auszubilden. Jede Position bringt zum Teil sehr unterschiedliche 
Anforderungen mit sich, weshalb nicht jeder Spieler auf jeder Position spielen kann. 
Gängige -- oder grundlegende -- Positionen und ihre Bezeichnungen sind beispielsweise 
der Torwart (\texttt{TW}), der Innenverteidiger (\texttt{IV}), der linke/rechte 
Außenverteidiger (\texttt{LV/RV}), das zentrale defensive Mittelfeld (\texttt{ZDM}), 
das linke/rechte Mittelfeld (\texttt{LM/RM}), das zentrale offensive Mittelfeld 
(\texttt{ZOM}) und der Stürmer (\texttt{ST}). Positionsbezeichnungen werden 
üblicherweise in Großbuchstaben angegeben und versuchen, die Rolle der Position 
widerzuspiegeln. \\ 
Im Rahmen des Spielzeitenplaners haben Nutzende die Möglichkeit, eine eigene Formation 
und Positionen -- basierend auf den oben erläuterten Konventionen -- zu erstellen. 

