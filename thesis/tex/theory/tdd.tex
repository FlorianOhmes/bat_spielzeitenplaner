
Test-Driven Development (TDD) -- zu deutsch: testgetriebene Entwicklung -- ist die 
Bezeichnung für eine methodische Herangehensweise der Softwareentwicklung. Die dieser 
Methode zugrunde liegenden Ideen, Prinzipien und Praktiken sowie ihr Nutzen sind bereits 
2003 von Kent Beck ausführlich beschrieben worden \cite{beck2003tdd}. Für detaillierte 
Ausführungen kann die zuvor genannte Monographie genutzt werden. Im Folgenden soll 
dennoch kurz auf diese wichtige theoretische Grundlage, die eine zentrale Rolle bei der 
Entwicklung dieses Projektes spielt, eingegangen werden. \\ 
Ein Zyklus des Test-Driven Development setzt sich aus den folgenden Phasen zusammen: 
\texttt{Red}, \texttt{Green} und \texttt{Refactor}. Diese Phasen sind der Reihe nach zu 
durchlaufen und der Zyklus so lange iterativ zu wiederholen, bis das Software-Projekt 
mit all seinen gewünschten Funktion schließlich vollständig realisiert ist. \\ 
Die Phase \texttt{Red} ist die erste Phase, deren Ziel es ist, einen Test zu schreiben, 
der fehlschlägt -- in der IDE also rot angezeigt wird. Der Inhalt des Tests beschreibt 
eine gewünschte Funktionalität, die in den folgenden Phasen dann im Produktiv-Code 
implementiert werden soll. Der Test schlägt also erwartungsgemäß fehl, da die 
Funktionalität zu Beginn noch nicht vorhanden ist. Die Farbe Rot bezieht sich dabei 
nicht nur auf fehlschlagende Tests, sondern auch auf eben jene Code-Schnipsel, die 
zunächst gar nicht kompilieren. Unter Umständen ist es also beispielsweise möglich, dass 
eine Klasse erst einmal erstellt, ein Konstruktor implementiert oder eine Funktion 
definiert werden muss, bevor mit dem Schreiben des Tests fortgefahren werden kann. \\ 
Auf die \texttt{Red}-Phase folgt die \texttt{Green}-Phase. Ist ein Test implementiert 
und fehlgeschlagen, so kann ein Stück Code geschrieben werden, dass gerade groß genug 
ist, sodass es diesen Test bestehen -- also grün werden -- lässt. Becks Fokus liegt 
dabei auf einem schnellen und unkomplizierten Umsetzen des gewünschten Features, bei dem 
jegliche Programmier-Sünden erlaubt sind. Diese können und sollen dann zu einem späteren 
Zeitpunkt ausgemerzt werden. Diese Herangehensweise kann als eine strenge 
Handhabung der TDD-Praktik bezeichnet werden. Im Gegensatz dazu ist es bei einer 
lockeren Handhabung des TDD durchaus denkbar, ein Stück Code direkt auf die 
beabsichtigte Weise zu implementieren, zum Beispiel im Falle routinemäßiger Aufgaben, 
die von Entwickelnden aufgrund ihrer Erfahrung bereits antizipiert werden können. \\ 
Die dritte Phase ist \texttt{Refactor}. Hier geht es um die Überarbeitung und 
Verbesserung des vorliegenden (Produktiv-)Codes, ohne dabei die Tests zu beschädigen. 
Nach einem \texttt{Refactor}-Schritt sind also nach wie vor alle bereits bestehenden 
Tests grün. Typischerweise werden beim \texttt{Refactoring} sämtliche Duplikationen 
im Code entfernt, Code-Schnipsel in Methoden ausgelagert oder ähnliche Schritte zur 
qualitativen Code-Verbesserung durchgeführt. \\ 
Test-Driven Development verfolgt das Ziel, sauberen Code zu produzieren, der 
funktioniert. Es stellt sicher, dass Software den definierten Anforderungen entspricht 
und fördert ein stärkeres Bewusstsein für den Zweck des vorliegenden Codes. Darüber 
hinaus kann TDD als eine Art Frühwarnsystem fungieren, das Alarm schlägt, wenn ein 
Test fehlschlägt, denn auf diese Weise können Fehler im vorliegenden Code frühzeitig 
erkannt und behoben werden. Außerdem kann TDD die Modularität der Software erhöhen und 
diese letztendlich wartbarer machen. 

